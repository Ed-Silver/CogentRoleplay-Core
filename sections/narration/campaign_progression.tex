If you have decided to play a campaign, choose a tier of play that feels appropriate for the story you want to tell, this should be done in concert with the party, but ultimately the narrator makes the final call.

\begin{itemize}
    \item If your adventurers are wide-eyed young adults leaving home for adventure for the first time, perhaps an **Initiate** tier game would be the most appropriate.
    \item If the party is a group of travelling adventurers, arriving in a new land from another place to take on quests and seek glory, **Adept** may feel the most appropriate.
    \item Or finally, the party may be the king's most trusted circle of knights, heroes in their own right, gathered together to face the greatest foe of an age. Then surely, a **Veteran** is the most suitable!
\end{itemize}

In the previous section, we discussed Story Arcs and Story Chapters. The campaign progression system uses those loose guidelines as benchmarks for how and when to award different bonuses. There are also modifications made to how many Attribute Points, Skill Points, Vocations and Disabling characteristics players may select at character creation. Finally, there is a special rule used in Initiate and Adept tiered games only, simply called a \textbf{“Career Milestone”}

\textbf{\textit{The following modifications apply to character creation:}}

\begin{displayquote}
    Starting Attribute Points: The number here replaces the amount of attribute points you get to assign in Fast Play games (Usually 2)

    Starting Skill Points: The number here replaces the amount of attribute points you get to assign in Fast Play games (Usually 12)

    Maximum Vocations: This number indicates the maximum number of Vocations you can select during character creation. (Eg: Chef, Wizard, Ranger etc)

    Disabling Characteristics: This number indicates the maximum number of Disabling Characteristics you may take at character creation. (Unrestricted in Fast Play games.)
\end{displayquote}

\textbf{\textit{The following skill caps affect player characters until they reach story milestones:}}

\begin{displayquote}
    Maximum Core Skill Points: This number indicates the maximum number of skill points you can invest in a single skill at character creation. (Unrestricted in Fast Play games)

    Maximum Vocation Skill Points: This number indicates the maximum number of skill points you can invest in a single Vocation or Vocation Skill (Including Combat Skills) at character creation. (Unrestricted in Fast Play games.)
\end{displayquote}

\section{Initiate Tier}

The journey of wide-eyed or inexperienced folk as they embark upon a campaign or adventure is often considered the most archetypal form of storytelling in TTRPGs. It can be immensely rewarding watching your character grow and soon find themselves able to easily thwart challenges that troubled them in the beginning sessions.

Initiate play is where a lot of long-form campaigns will start, due to its very limited distribution of skills and its rapid growth. Players will and should feel weaker or more restricted, but not more so than common folk. The Narrator should take extra care to ensure that early challenges are appropriate for lower-skilled characters. Dice pools will likely be no larger than 6 at maximum, so afford suitable opportunities for roleplay to carry the day.

\textbf{\textit{The following modifications apply to your character and how they can advance. These are listed as the campaign progresses as your character becomes stronger, and wiser and hits milestones.}}

\begin{center}
    \begin{xltabular}{\textwidth}{|l c|} 
        \hline 
        \textbf{Modifications} &  \\
        \hline
        Maximum Core Skill Points & 2 \\
        Maximum Vocation Skill Points & 2 \\ 
        \hline
    \end{xltabular}
\end{center}

\textbf{\textit{The following modifications apply to character creation, as soon as the campaign progresses, these restrictions no longer apply to characters (they can learn new vocations, and may acquire new disabling characteristics if relevant). The values for starting attributes and skills replace the ones found in \hyperref[part:character_creation]{Part \ref{part:character_creation}: Character Creation}:}}

\begin{center}
    \begin{xltabular}{\textwidth}{|l c|} 
        \hline 
        \textbf{Modifications} &  \\
        \hline
        Starting Attribute Points & 1 \\
        Starting Skill Points & 6 \\ 
        Maximum \# of Starting Vocations & 1 \\
        Maximum \# of Disabling Characteristics & 1 \\
        \hline
    \end{xltabular}
\end{center}

\subsection{Initiate Progression Guidelines}

\textbf{First Taste of Glory:} Initiate campaigns should reward players with a skill point upon the conclusion of their first session or quest. Unlike most progression points, this does not have to be a significant enough event to call it a chapter, even slaying pesky rats in a cellar will do. Introduce the idea of growth early, and players will be engaged to keep growing.

\subsection{Acts and Chapters in Initiate Campaigns}

Due to the maximum cap on skill points that can be invested into skills, it is safe to award skill points relatively regularly, with players broadening their skill set rather than becoming “best of class” inappropriately early (to make sense in the story).

An Arc should have roughly 6-8 chapters within it, and each chapter should reward a skill point to spend upon completion. Ensure you also award a destiny point at the completion of each chapter as well.

\begin{displayquote}
    At the end of the FIRST arc, increase the maximum CORE skill points and VOCATION skill points to 3.

    At the end of the SECOND arc, increase the maximum CORE skill points to 4. Award an ATTRIBUTE point to all players.

    From the THIRD arc onwards, award skill points every second chapter, rather than every chapter. Continue to award a destiny point every chapter.

    At the beginning of the FINAL arc, award an ATTRIBUTE point to all players.
\end{displayquote}

\textbf{Career Milestone:} At any point after the end of the \textbf{SECOND} arc, it can be exceedingly rewarding to involve individual character life motivations into the story. Come up with an individual quest or goal for each player that represents them entering the elite tier of their chosen life path. As each player completes this quest, increase their maximum \textbf{VOCATION} skill points to \textbf{4}.

Optionally, this can be done for each vocation a player has, with the maximum skill point cap lifting for the vocation they complete the quest for.

\textit{Example:}

\begin{displayquote}
    The troubled archer, Melesandra, has been plagued by nightmares of the beast that killed her family when she was young. After adventuring throughout the campaign, the players reach midway through the third story arc. Unexpectedly, in a small village, a group of townsfolk gossip about a great beast that matches the description of the one that took Melesandra's family. If she were to find it, and slay it… She would have conquered her demons and proven herself as an expert archer.
\end{displayquote}

Should you choose not to use Career Milestones, instead increase all players' maximum \textbf{VOCATION} skill points to 4 at the end of the \textbf{SECOND} arc, along with \textbf{CORE} skills.

\section{Adept Tier}

\textbf{\textit{The following modifications apply to your character and how they can advance. These are listed as the campaign progresses as your character becomes stronger, and wiser and hits milestones.}}

\begin{center}
    \begin{xltabular}{\textwidth}{|l c|} 
        \hline 
        \textbf{Modifications} &  \\
        \hline
        Maximum Core Skill Points & 3 \\
        Maximum Vocation Skill Points & 3 \\
        \hline
    \end{xltabular}
\end{center}

\textbf{\textit{The following modifications apply to character creation, as soon as the campaign progresses, these restrictions no longer apply to characters (they can learn new vocations, and may acquire new disabling characteristics if relevant). The values for starting attributes and skills replace the ones found in \hyperref[part:character_creation]{Part \ref{part:character_creation}: Character Creation}}}

\begin{center}
    \begin{xltabular}{\textwidth}{|l c|} 
        \hline 
        \textbf{Modifications} &  \\
        \hline
        Starting Attribute Points & 2 \\
        Starting Skill Points & 12 \\ 
        Maximum \# of Starting Vocations & 2 \\
        Maximum \# of Disabling Characteristics & 2 \\
        \hline
    \end{xltabular}
\end{center}

\subsection{Adept Progression Guidelines}

\textbf{Local Renown:} Adept adventurers have already made a name for themselves, in the starting area of the campaign, players should have a reputation. They are more likely to be dealt with respectfully, and sought out for help.

\textbf{Starting Supplies:} Typically, Adept adventurers should have all the tools they need to go forth and adventure, horses and a wagon in a medieval campaign, or perhaps a small shuttle craft in a sci-fi adventure.

\subsection{Acts and Chapters in Adept Campaigns}

An Arc should have roughly 6-8 chapters within it, and each chapter should reward a skill point to spend upon completion. Ensure you also award a destiny point at the completion of each chapter as well.

\begin{displayquote}
    At the end of the FIRST arc, increase the maximum CORE skill points 4.

    At the end of the SECOND arc, award an ATTRIBUTE point to all players.

    From the THIRD arc onwards, award skill points every second chapter, rather than every chapter. Continue to award a destiny point every chapter.

    At the beginning of the FINAL arc, award an ATTRIBUTE point to all players.
\end{displayquote}

\textbf{Career Milestone:} At any point after the end of the \textbf{FIRST} arc, it can be exceedingly rewarding to involve individual character life motivations in the story. Come up with an individual quest or goal for each player that represents them entering the elite tier of their chosen life path. As each player completes this quest, increase their maximum \textbf{VOCATION} skill points to \textbf{4}.

Optionally, this can be done for each vocation a player has, with the maximum skill point cap lifting for the vocation they complete the quest for.

Should you choose not to use Career Milestones, instead increase all players' maximum \textbf{VOCATION} skill points to 4 at the end of the \textbf{FIRST} arc, along with \textbf{CORE} skills.

\section{Veteran Tier}

\textbf{\textit{The following modifications apply to your character and how they can advance. These are listed as the campaign progresses as your character becomes stronger, and wiser and hits milestones.}}

\begin{center}
    \begin{xltabular}{\textwidth}{|l c|} 
        \hline 
        \textbf{Modifications} &  \\
        \hline
        Maximum Core Skill Points & 4 \\
        Maximum Vocation Skill Points & 4 \\
        \hline
    \end{xltabular}
\end{center}

\textbf{\textit{The following modifications apply to character creation, as soon as the campaign progresses, these restrictions no longer apply to characters (they can learn new vocations, and may acquire new disabling characteristics if relevant). The values for starting attributes and skills replace the ones found in \hyperref[part:character_creation]{Part \ref{part:character_creation}: Character Creation}}}

\begin{center}
    \begin{xltabular}{\textwidth}{|l c|} 
        \hline 
        \textbf{Modifications} &  \\
        \hline
        Starting Attribute Points & 2 \\
        Starting Skill Points & 18 \\ 
        Maximum \# of Starting Vocations & 3 \\
        Maximum \# of Disabling Characteristics & 3 \\
        \hline
    \end{xltabular}
\end{center}

\subsection{Veteran Progression Guidelines}

Veteran campaigns are slightly harder to progress due to already commencing near the pinnacle of human capacity. So exclusively in Veteran campaigns, it can be an interesting option to play with using the maximum number of destiny points as a new method of progression.

\begin{itemize}
    \item It can be tempting to allow characters to grow beyond the restrictions within the core rules for humans, but we advise against it. Save those options for supernatural, magical or technological enhancements that can be obtained in-game.
    \item If your veterans are superheroes or similar, that is of course a different story. And it is up to the narrator to determine what those stat modifications will look like. Alternatively, you can look at some examples in Chapter 7: Supplementary content, for ideas.
\end{itemize}

Ask yourself and your players why seasoned and top-of-their-career veterans are setting out on an adventure, and what that means for the narrative.

\textbf{Reputation Precedes them:} Typically veteran adventurers are well known by friends and foes alike in a wide area, they should be treated as such by any NPCs that would know their names.

\textbf{Glory Days: (Are over?)} An optional rule for Veteran campaigns only. With much of their glory and success behind them, Veterans adventurers start the campaign as if they had “used up what luck fate had for them” They have a maximum destiny point pool of ONE. However, this maximum will grow every arc until it exceeds the usual limit of THREE, as fate proves to them they are the chosen heroes.

\subsection{Acts and Chapters in Veteran Campaigns}

An Arc should have roughly 6-8 chapters within it, and each chapter should reward a skill point to spend upon completion. Ensure you also award a destiny point at the completion of each chapter as well.

\begin{displayquote}
    At the end of the FIRST arc, award an ATTRIBUTE point to all players. If using Glory Days rule, increase their maximum destiny points to 2.

    From the SECOND arc onwards, award skill points every second chapter, rather than every chapter. Continue to award a destiny point every chapter.

    At the end of the SECOND arc, if using Glory Days rule, increase their maximum destiny points to 3.

    At the beginning of the FINAL arc, award an ATTRIBUTE point to all players, if using Glory Days rule, increase their maximum destiny points to 4.
\end{displayquote}
