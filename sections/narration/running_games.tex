\section{Running a One Shot}

One of the simplest ways to get started Narrating is to run a self-contained single-session adventure. Often using Cogent Roleplay, this can be further simplified by diving into a familiar world or setting with friends! Using a familiar setting allows all the players at the table to innately understand a lot of the rules around how the world works, and lightens the load on the narrator. A one-shot is simply a short one-off story, it doesn’t always finish in a single sitting, but usually, that is the intent. Characters made for one-shots aren’t used later, so they can be a great chance to try new things and practice new avenues of roleplay for everyone at the table. During a One Shot, we recommend using the \textbf{Fast Play} rules and creating characters exactly as described earlier in the rules. Players typically won’t \underline{gain} skill points or even destiny points \underline{during} a One Shot, however Narrators should always reward exceptional roleplay with a destiny point, even in one-and-done settings.

\section{Running a Campaign}

Running a Campaign can be one of the most singularly rewarding TTRPG experiences one can have. The friends and stories made at the table can very much be lifelong, and the journey your characters go on will be remembered for years to come. As the Narrator, you are in the exciting position of being able to frame that story, the players act out and respond to all the mystery, intrigue or drama that you can throw at them. Campaigns can run for a short time; perhaps a half dozen sessions, but they can also run for massively extended periods. Campaigns spanning years or even decades are not unheard of, and the level of investment in the world is incomparable to one-shots. With this in mind, it is important to have, at the very least, a general idea of the overarching plot and story you wish to tell. Then loosely we can break up that plot into story acts.

A story act is a period of time in the campaign where the players encounter a larger issue or seek a goal and resolve it over the course of the game. This issue or goal should be significant, and scaling to the peak of a story act should be a significant occasion. Story acts can also be known as arcs, or perhaps seasons if you think of a campaign like a TV show. There are many different configurations to the overall story structure, but a very common one is the concept of “Three Acts”.   In the first act, players face some obstacles, both outside and inside the party, but overcome their local threats and come into their own. In the second act, they begin strong, but slowly things turn dire as a nasty villain’s plot becomes known. By the end of the second act, often characters have suffered a major loss or setback, and hope seems lost. The third act begins with our characters finding resolve, and challenging the enemy, ultimately cresting the heights of their potential and overcoming their greatest challenge. This is just one example of a campaign as it plays out as a representation of the most archetypal ‘three-act story’ you can craft. You do not need to craft your stories like this, but it offers a guideline or rough concept of how story arcs flow and each should build up to a satisfying conclusion. (Satisfying can be negative or positive for the player characters, so long as it serves the story)

Within a story arc there are multiple chapters or episodes. Traditionally in tabletop RPGs, these do not represent play sessions but rather quests or lesser plot events. You may cover a story chapter in a single session, or it may take three or four. These chapters serve as small milestones as your characters learn new things and move out into the world, slowly working towards the goal of the story's current arc (or season).

These terms are how we frame character advancement in games of Cogent Roleplay. There are no experience points to gain, instead, milestones players reach that show they have advanced their skills and talents.

\section{Ways to Play}

Several times throughout the Cogent Roleplay rules we have referred to Fast Play and Campaign Play, here in the narrator section we will go into depth about what it means.

\textbf{Fast Play} is simple, at its core, \textbf{Fast Play} is the basic form of the cogent rules. In \textbf{Fast} \textbf{Play} games, you can use everything presented here at face value. It is intended for shorter stories, one-shots or settings that might take place over a short space of time in the world, where character advancement doesn’t fit the story Narrators and players want to tell.

\textbf{Campaign} \textbf{Play} is a different beast entirely. \textbf{Campaign} \textbf{Play} is designed to allow for a rewarding growth of characters throughout a longer format storytelling game. It covers everything from episodic encounters with large time jumps to day-to-day heroes’ journey campaigns that could span years. \textbf{Campaign} \textbf{Play} applies restrictions to certain rules as well as guidelines on how often to reward players with different resources and skills. Nothing in \textbf{Campaign} \textbf{Play} is designed to change any of the core rules or features of Cogent Roleplay, instead, it offers a framework to allow players to feel a sense of growth and progression throughout their journey.

As a guideline, \textbf{Campaign} \textbf{Play} is split into three tiers, each representing a starting point for your characters. These tiers are \textbf{Initiate}, \textbf{Adept} and \textbf{Veteran}. Within each tier, stat distribution and reward recommendations differ. As a general guide, \textbf{Initiate} tier games start with weaker characters that progress more often, whereas \textbf{Veteran} games are the opposite.