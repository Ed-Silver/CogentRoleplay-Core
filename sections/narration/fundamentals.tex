In the context of Cogent Roleplay, the Narrator is the storyteller. They perform a function similar to that of a Dungeon Master or Game Master in other game systems. However, Cogent Roleplay focuses more heavily on group storytelling than many other game systems, so in a way, the role of the narrator is one of facilitating that story through the creation of the majority of the worlds and characters the players will encounter and explore.

One way to look at it is that the Narrators set the scene, fill the scene with extras and then prepare the stage for the actors to perform their play. As a narrator, you facilitate gameplay, role-play as various NPCs, arbitrate disputes and establish a plot that the players engage with. It is a role with a very wide range of responsibilities, but do not let that dissuade you. The rewards for being the narrator are numerous, and it is an extremely satisfying role to fill.

\section{How do you become a Narrator?}

In general, there are more people who desire to be players than narrators of stories, so as an aspiring Narrator it should not be too hard to convince people to let you give it a shot. All you need to do is have an idea of a story you wish to tell, or a setting you wish to play in, and pitch it to your players. From there, a lot of it is learned by doing. There is no simple method to becoming an excellent narrator for the simple reason that what everyone values in a narrator is different. A legendary narrator in one playgroup may be considered a bad fit for a different one. This is why seeking to establish a good dynamic at the table as players of a game is critical to having an easier time narrating games. While there are no catch-all solutions, here are some simple tips to make your first game as a narrator easier:

\begin{itemize}
    \item Talk to your players: Discuss the style of game they wish to play, and ensure everyone is happy with the choice. (eg: A roleplay-heavy dark story, A crime thriller, political intrigue or combat dungeon crawls)
    \item Facilitate a group character creation session: It will make your first games infinitely easier if the party isn’t constantly on the verge of fracturing and splitting. Run a group character-building session and encourage bonds between player characters that reach into their history, cementing them as a group with a logical reason for being together.
    \item Don’t compare yourself to others: It is important to remember that every journey is different and every story is worth telling in its own way. Focus on having fun as a group, and don’t get caught up on perceived performance.
    \item Offer controlled choice early: Bound your first sessions into smaller sandboxes. Limit player movement and choice but try not to limit their agency. You can control the options, but don’t force the decisions.
    \item Be ready to improvise and don’t cement too much: As a Narrator, it can be beneficial to have solid ideas of plot beats and character goals, but it is important to allow them to be flexible enough that the story can survive and sculpt around exciting player developments.
\end{itemize}