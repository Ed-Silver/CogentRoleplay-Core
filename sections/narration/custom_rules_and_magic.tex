\section{Homebrew}
One of Cogent Roleplay's greatest strengths is its simplicity and its adaptability. The core rules presented here are designed to be as flexible as possible and applicable to any setting you desire. However, should you desire more specific flavour and depth for a setting, you are encouraged to invent your own custom rules and homebrew concepts for that setting.

An example setting has been added to \textbf{Supplementary Content}, for you to use or use as inspiration, and more may be added in the future.

\section{Metahuman (Magic and Superpowers)}

The Vocation skill category is also to be used for supernatural, or metahuman abilities if the setting you’re playing in has any. (For the sake of simplicity, from this point onward any type of magic, superpower or sci-fi enhancement will simply be referred to as “magic”, and when referred to, it applies also to all supernatural abilities that could be added into any setting.)

Depending on the setting, the Narrator may have different options in place for how magic works, however regardless of the system your skill points will be invested in a Vocation to manage those powers. A rough example of what that might look like is shown below:

The following must be considered when a Narrator decides how the magic works in the setting:

Can anyone attempt to perform a spell even without the specific combat skill in the same way anyone can pick up a sword and try to use it? Remember that in both cases with the sword and magic, having no proficiency with them would result in many failed rolls. Can only people with at least 1 combat skill point in the magic use it?

Do you want magic to be more powerful than normal combat skills, or have it balanced? 

How encompassing do you want the magic to be? Is there a single magic proficiency or is it broken down into categories such as Earth, Air, Fire and Water, Sorcery, Wizardry, Witchcraft, Druidism, Necromancy, or Holy?

Is there a limit to the magic’s use? Can a player only cast a certain amount of spells a day or per combat encounter?

\section{A Few Words on Magic and Weapons:}

As was explained in the combat chapter, every weapon receives a weapon bonus depending on its type. This bonus stays when making any roll that uses weapon proficiency and generally results in more dice to roll with when using proficiencies than with skills. It’s up to the narrator if they wish to add weapon-type bonuses to magic proficiencies.

For instance, using a wand might grant a +1 die while using a staff might grant a +2. Alternatively, if the narrator does not wish to add weapon bonuses to magic proficiencies they might decide that the magic simply cannot be used without a wand or staff.

If weapon bonuses are added to magic proficiencies, the magic will then have the same amount of dice as other combat/weapon proficiencies – this might seem balanced but it may not be – It can potentially make magic more powerful than common combat/weapon proficiencies depending on the setting. A player can do far more with telekinetic powers than with a sword.

Ultimately, how this is balanced is up to the Narrator or the specific setting games that are being played in. Remember it is always critical to consider how game balance affects player experience, you don’t want your wizard outshining everyone else to such a degree that they feel useless. In a fast play Game, it might be a lot of fun to roleplay as indestructible superheroes or Greek Gods. In the case of a campaign it can be very frustrating for players to be the least powerful member of the group. Conversely, if a player is too powerful they may become bored by the lack of a real challenge.

A very simple solution to this is to split magic up into its subcategories and insist that players must be trained (have a skill point in a relevant vocational skill or combat skill) in order to attempt to use magic. This ensures magic users have access to powerful abilities, but require large sums of their skill points to be invested in one area, making them significantly less useful outside of those situations.

In the chapter: supplementary Content, we have provided an example magic system that can be used or adapted into various settings – or used as a model on which you may craft your own (see \textbf{Example Magic System}.)