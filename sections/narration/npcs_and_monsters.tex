\section{NPCs}

Non-player characters are vital to stories told in any TTRPGs, they can be so memorable that players mourn their deaths and cheer their victories. They are, essentially, the direct representatives of the narrator’s fantasy world. While enthusiastic narrators may attempt to ‘stat out’ every NPC they create, we caution against investing all your time into areas such as that. While occasionally, a major and consistent NPC may need a full character sheet, more often than not an NPC can be recorded in simpler terms. A name, a general disposition, and a short outline of who they are and where they’ve been. It can also be good to give them ‘rough dice pools’ – such as “Intelligence skills, 5D6” if you expect it to come up.

When preparing enemies for combat, it is important to remember that player characters represent exceptional and standout individuals. Even \textbf{‘Initiate’} campaign characters start with 1 attribute point, whereas the average human has 0/0/0 in their attributes.

As a rough guideline, an average person in an average position has 0 attributes, and 2 in their vocational skills. This means a town guard would likely have a dice pool of Base Three, plus 0 for attributes, plus two for a vocational weapon skill such as bows or swords, and then any weapon bonuses.

Elite NPCs, or guard veterans/captains, would likely have a single attribute point, and 3 points in their vocational skills.

Only exceptional NPCs have two or rarely more attribute points in a standard setting, with 4 points in their relevant vocation/weapon skills.

\textit{Examples:}

\begin{center}
    \begin{tabular}{|p{0.15\textwidth} p{0.22\textwidth} p{0.22\textwidth} p{0.3\textwidth}|} 
        \hline 
        \textbf{NPC} & \textbf{Attributes} & \textbf{Skills} & \textbf{Vocations} \\ 
        \hline
        Thug & 
            $\bullet$ Strength 1 & 
                $\bullet$ Athletics 2 \newline 
                $\bullet$ Endurance 2 & 
                    $\bullet$ Mussel 2 \newline
                    \hspace*{.7em} $\circ$ Small Weapons 2 \newline
                    \hspace*{.7em} $\circ$ Medium Weapons 1 \newline
                    \hspace*{.7em} $\circ$ Intimidation 1 \\
        \hline
    \end{tabular}
\end{center}

\section{Beasts and Monsters}

Cogent roleplay has been designed in such a way that its core skills can be used by anything, human or otherwise. And the vocation system allows for creatures to take vocations and unique skills, with points assigned to them as usual. In a simple setting, it would allow for a creature such as a Lion to have the Vocation “Plains Predator” with skill points invested in “Teeth and Claws” for its weapon attacks. Beasts and monsters are the exceptions to the rule that combat skills must align with a weapon type. While it is recommended that you use the weapon types and their combat bonuses as a template, you are free to make up whatever combat skill feels appropriate for a creature. In more esoteric settings, the narrator can use these vocational skills to represent inhuman abilities that they have made part of their campaign setting.

\textit{Examples:}

\begin{center}
    \begin{tabular}{|p{0.15\textwidth} p{0.22\textwidth} p{0.22\textwidth} p{0.3\textwidth}|} 
        \hline 
        \textbf{Monster} & \textbf{Attributes} & \textbf{Skills} & \textbf{Vocations} \\ 
        \hline
        Korinkian Boreworm & & & 
            $\bullet$ Ambush Predator \newline
            \hspace*{.7em} $\circ$ Rock Tunnelling 3 \newline
            \hspace*{.7em} $\circ$ Rock Mimicry 2 \newline
            \hspace*{.7em} $\circ$ Grinding Teeth 3 \\
        \hline
    \end{tabular}
\end{center}