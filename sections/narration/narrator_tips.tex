Many issues can arise at the table when running a TTRPG, and more often than not, it falls to the narrator of the game to resolve them. As players become heavily invested in their characters, emotions can run hot as things do not go as they expected. Beyond just interpersonal management, certain campaigns can cover heavy themes and go to dark places. Sometimes, exploring these aspects of the human psyche and experience can be immensely rewarding, learning experiences, but all things have an appropriate time and place. The following are some general narrator suggestions that can be used in your games, they aren’t mandatory by any means but offer some things to think about.

\section{Honesty and Conversation are the Key}

The golden rule when starting out a campaign is to have discussions with your players about what they want and what they can expect. It is perfectly fine to ask them in broad terms what style of game they want to play and to talk out as a group about what will be the most enjoyable for everyone.

During this discussion, asking questions of your players can help everyone at the table get an understanding of each other’s expectations. Set boundaries and restrictions. These aren’t right or wrong, they are all personal choices. Here are some examples of these ideas as they relate to roleplay that could come up in this sort of conversation:

\begin{displayquote}
    No explicit content
    No drug references
    No in party "loot hoarding" or theft. (One might argue that it is "in character" to do so, but if it will upset or frustrate a player consistently in real life, then you shouldn't do it.)
    Group unity (Players agree to create characters that naturally work together and don't fight often)
\end{displayquote}

There are endless discussions to be had here, and it is much better to nip a problem in the bud before it arises than have to manage the fallout.

\section{Mature Themes}

When dealing with adult themes of any nature, you absolutely must be certain that you and the players are all ok with including them. Discuss this with your players, and discuss where the boundaries are. Even with such discussions, as a Narrator, you should offer options for your players to back out of any situation that makes them feel uncomfortable (even if they indicated it was ok prior)

\textit{Example:}

\begin{displayquote}
    Red Card": Players and Narrator all have a red card they can deal to the table as soon as something becomes an issue for them. The scene ends. No questions asked. Roleplay picks up from the next appropriate location.

    "Rewind": Players and Narrator can freely "Rewind" a situation by dealing a similar card. This can allow a narrator to choose a different approach to the situation. The previous encounter prior to the rewind is stricken from the record, and considered non-canon. Obviously this should be used for player protection, not character protection. This isn't a "get-out-of-character-consequences free card".
\end{displayquote}

\section{Player Disputes}

Occasionally, players may feel that they have been treated unjustly by the game. They may question decisions or outcomes that occurred at the table. When this happens, it is important to listen to them and take their views into consideration. While ultimately the narrator makes the final call on arbitration, considering the outcome of what has happened in-game, and how the players feel about it. Sometimes, the player may be out of line, angry or being unfair, in these cases, it is understandable to let a ruling stand. But remember that TTRPGs at their core are a group of friends playing make-believe with each other, and if one player’s enjoyment has been completely taken away, it isn’t a good experience for everyone.  A potential solution to this is to ask the table how they feel it would be fair to resolve the issue. For example, if a player character died and they felt they had no chance to avoid the death, consider explaining the situation, and asking the group what they feel would be a better outcome for the story, perhaps their character is instead taken prisoner, or severely injured and in a coma. There are ways that can avoid an element of trauma while still ultimately requiring the player to create and play with a new character, but now the added story element of their last being missing or out of action can fuel story beats rather than hinder them.

\section{Commerce}

To help make the experience of buying/selling dynamic for players, a narrator might consider allowing the cost of the desired item to ‘drop’ by one level of value if they succeed in a ‘persuasion’ or ‘deception’ check vs. a vendor. Likewise, the item may also become more difficult to obtain if such a dice check is failed.

It is also worth considering what items are available and where. A small town in a medieval setting may have a farrier who can smith up some horseshoes, but it is doubtful that they would have the facilities or artisans to create full-plate armour of the highest quality. Introduce magical items into the mix and it complicates things further. Having towns with notable trades in your setting can make them feel richer and more memorable, as well as building stories within those areas easier. An example of this could be a town built by a cliffside with a rich vein of gemstones running through it offering a lucrative trade of such goods, while they may be far more scarce elsewhere.

A narrator may also decide to allow the players to use their destiny points in making purchases more attainable if this can be justified through narrative in a satisfying way. That being said, keep in mind that destiny points scale at a level of 1:1, while commerce points scale at a ratio closer to 1:10, therefore a narrator may decide it may cost a single destiny point for a level 3 item to be lowered to a cost of 2 commerce points to a player, but it might cost 2-3 destiny points to bring down the cost of an item valued at level 5-6.