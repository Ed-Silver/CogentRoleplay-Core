Something crucial in Cogent Roleplay is how you can treat success and failure.

In Cogent, success can be seen as fairly straightforward, if you meet the desired challenge level your action succeeds as expected. But there is scope to increase the effectiveness of the success if the player beats the \textbf{CL} of a task by a significant amount. This is often used in scaling checks, most commonly used with the perception skill.

The Narrator can opt to set a task with a minimum \textbf{CL}, where additional successes achieve greater results. 

\begin{center}
    \underline{This can be any \textbf{CL} they wish, but if they do not set a \textbf{CL}, it is assumed to be 1.}
\end{center}

\textit{Example:}
\begin{displayquote}
    A player asks to look around a crime scene to get their bearings. The narrator sets the CL at 1, with a total score of 1 ensuring the player gets a basic rundown of the scene, but the more they beat the CL, the deeper and more significant the information they find becomes.
\end{displayquote}

In many ways, these scaling checks can be seen as the inverse of reflexive actions. They usually apply to perception, but the narrator may choose to award similarly beneficial results to other types of checks.

\textit{Example:}
\begin{displayquote}
    A player asks to pick a lock. The narrator sets a CL of 3. The player has a dice pool of 6 and amazingly scores 6 successes. The Narrator declares that the player so deftly cleared the lock that they realized they are familiar with the exact maker of the locks in the jail, and can confidently pick any more that remain. They then state the player automatically passes all further checks related to picking that type of lock in this location.
\end{displayquote}

When it comes to failure, most often people think of the Literal Failure of the task – Failing to pick a lock means… well, you failed to pick a lock. Dice rolls are most often interpreted this way, and are standard practice in many PnP RPGs.

In addition to Literal Failure, Cogent Roleplay encourages the use of two methods of story progression, Narrative Failure and Failing Forward.

Only rarely is a path of action truly misguided enough to lead to no narrative outcomes, and this is the area that literal failure should apply. Sometimes, a magical runic lock just needs the correct magical phrase to breach, and nothing else will do. These can be important narrative foils to get players moving toward concrete goals and should serve to advance the story by directing player focus rather than grinding it to a halt. If you find a challenge is grinding the game to a halt unintentionally, it is time to move on to Narrative Failure or Failing Forward.

\section{Narrative Failure} \label{sec:narrative_failure}

Narrative Failure doesn’t mean that the action itself failed, but rather the desired outcome is not obtained.

\textit{Example:}

\begin{displayquote}
    If a player is trying to sneak into a walled city and fails the assigned CL, a Narrative Failure might mean that they still climbed the wall (as intended), but are greeted by a guardsman at the top of the wall (… clearly not the desired outcome!) It might also mean when they crest the wall they knock off a loose stone which crashes to the ground below, making a VERY loud noise.
\end{displayquote}

If the player’s roll produces only losses and no wins, this can be considered a critical failure. The narrator might have the wall fall over while the character is clinging to it, crushing the king who just so happened to be walking by on the other side in front of countless witnesses.

Narrative Failure may provide interesting avenues of storytelling, and add loads of fun. Of course, sometimes Narrative Failure just doesn’t fit or a Literal Failure makes more sense. Alternatively, there is a third avenue open to moving forward.

\section{Failing Forward} \label{Failing Forward}

In instances where literal failure will cause the story to grind to a halt and narrative failure seems inappropriate, it is time to Fail Forward.

Failing forward differs from the other two methods of progression because both the intended action failed AND narrative failure didn’t allow them to achieve their desired outcome even with downsides. So how do we move forward?

Failing forward is what happens when the negative or failed outcome itself leads towards a NEW path forwards. In the example of the attempted wall climb above, if for example, it seemed so unrealistic that the party could climb the wall at all, and their roll was that poor, perhaps they are caught at the base of the wall by the guards who arrests them on suspicion and takes them into custody in the guardhouse.

The guardhouse just so happens to be on the other side of the wall.

Now the situation is dire, and a bad outcome has occurred, but the narrator has replaced one problem the players were unable to resolve with a new equally interesting narrative situation and a new set of challenges. Should the players escape their custody, surely new consequences await, but they will be free on the other side of the wall they had been trying to cross.