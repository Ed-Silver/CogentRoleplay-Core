When a player wishes to undertake a task within the game against something that cannot intentionally resist, it is referred to as an Action. Examples of this include:

\begin{itemize}
    \item Climbing a wall.
    \item Balancing on a tightrope.
    \item Recalling information.
\end{itemize}

This action is completed by rolling your character-relevant **Dice Pool** against a static \textbf{‘Challenge Level’}, or \textbf{‘CL’}.

When a player character chooses to make an action within the game, the Narrator can assign them a CL based on the difficulty of what they will attempt to perform, as well as indicating which skill should be used for the task – the more difficult the task, the higher the \textbf{CL}, ranging from \textbf{1} (a common task) to \textbf{8} (inconceivable!).

With the challenge level set and the skill indicated, a player can form their dice pool as described in the previous section. Once formed, they roll their dice pool in an attempt to beat the \textbf{CL} of the task.

The characters’ statistics (see Character Creation’) will determine the number of dice that can be rolled against the CL. Each dice represents a 50/50 chance to gain a \textbf{win} or \textbf{loss} towards the \text{CL} (unless a Destiny Point is used, see Destiny Point.)

\begin{figure}[H]
    \includegraphics[width=8cm]{images/placeholder}
    \centering
    \caption{Image Placeholder.}
\end{figure}

\textit{On a traditional 6-sided dice (or ‘D6’) any dice roll that shows a 1-3 is considered a loss, while any dice that shows a 4-6 is a win.}

To pass the \textbf{Challenge Level} you must roll enough \textbf{wins} equivalent to, or greater than, the set \textbf{CL}.

\textit{Example:}

\begin{displayquote}
    Your character wants to climb a wall and the Narrator assigns this task a \textbf{CL} of \textbf{3}. You have five dice to roll which means at least three dice must achieve a \textbf{win} to succeed against the \textbf{CL}.
\end{displayquote}

%\begin{figure}[H]
%    \includegraphics[width=8cm]{images/placeholder}
 %   \centering
  %  \caption{Image Placeholder.}
%\end{figure}

\begin{figure}[H]
    \centering
    \begin{minipage}[b]{0.4\textwidth}
        \includegraphics[width=\textwidth]{images/placeholder}
        \caption{\textit{Narrative Failure of -1}}
    \end{minipage}
    \hfill
    \begin{minipage}[b]{0.4\textwidth}
        \includegraphics[width=\textwidth]{images/placeholder}
        \caption{\textit{Narrative Success of 0}}
    \end{minipage}
\end{figure}

The more dice a player has to roll against a **CL**, the higher the chance they will succeed.

SEE CHEAT SHEET REFERENCE: CHSH - Challenge Level (CL) Difficulty Scale

\section{Assigning CL and Skills} \label{sec:assigning_cl_and_skills}

For every action the players wish to perform, it’s the Narrator’s job to assess each request and judge how difficult such a task would be. As well as this, it is the Narrator’s responsibility to set the skill that the player will be using for this challenge. While often a player will suggest a reasonable skill and the narrator will be happy to go with that choice, the final decision of the appropriate skill is up to the narrator.  

Often, the skill selection will occur organically at the table.

\textbf{Challenge levels} are assigned to any task that has a reasonable chance of failure. The more difficult the task, the higher the \textbf{CL}. These \textbf{CL} ratings are classified as follows:

\begin{center}
    \begin{xltabular}{0.5\textwidth}{|l|c|} 
        \hline 
        \textbf{Task} & \textbf{Challenge Level} \\ 
        \hline
        Common Task & CL 1 \\
        Uncommon Task & CL 2 \\
        Specialized Task & CL 3 \\
        Difficult Task & CL 4 \\
        Extremely Difficult Task & CL 5 \\
        Unrealistic Task & CL 6 \\
        Virtually Impossible Task & CL 7 \\
        Inconceivable Task & CL 8 \\
        \hline
    \end{xltabular}
\end{center}

Using these classifications, they choose an appropriate skill alongside the player and assign the \textbf{CL}.

In this example, a player is chasing a thief across rooftops in a busy city. The narrator describes the scene, a wide road separates the player from their quarry, the gap is large, but the street is a bustling bazaar littered with shade sails and guide ropes.

\begin{displayquote}
    \textit{\textbf{Player:}} "I would like to cross from my rooftop to the other side of the street"

    \textit{\textbf{Narrator:}} "Ok, how would you like to go about this?"

    \textit{\textbf{Player:}} "I wish to get a run-up and leap!"

    \textit{\textbf{Narrator:}} "Sure, you charge at the ledge and jump! Make a CL 4 Athletics check to cross the large gap"
\end{displayquote}

\textit{Alternatively:}

\begin{displayquote}
    \textit{\textbf{Player:}} "Can I balance on guide ropes and dart between shade sails to get to the other roof with my acrobatics?"

    \textit{\textbf{Narrator:}} "Certainly! Make a CL 4 acrobatics check to keep your balance as you cross."
\end{displayquote}

In this example, the challenge level was roughly equivalent, but in this negotiation process we see how multiple skills can be used to achieve the same result, and often the CL for each approach will be different depending on the skill use.

Below are some examples of CL’s a Narrator might choose to assign to various tasks. Each Narrator is entitled to make their own judgements on how difficult a task is, and CL’s will often vary between Narrators, campaigns, task circumstances and game sessions.

\begin{center}
    \begin{xltabular}{0.7\textwidth}{|l|c|} 
        \hline 
        \textbf{Example} & \textbf{Challenge Level} \\ 
        \hline
        Ride a horse without a saddle & CL 3 \\
        Climb a rope 20 meters in the rain & CL 4 \\
        Override the console of a basic spacecraft & CL 4 \\
        Do a triple back-flip & CL 5 \\
        Catch a flying arrow & CL 6 \\
        Lift a car & CL 7 \\
        \hline
    \end{xltabular}
\end{center}

A narrator may be tempted to assign a higher or lower CL to a more capable character because such tasks would be easier to them based on their skills. This is not necessary as CL’s should always be assigned based on how difficult the task would be to a regular person in that setting. It is a character’s skills (or lack thereof) that will affect their potential to succeed in action – not an altered CL.

However, CL can differ based on the approach to action, even if the outcome is the same. A player who attempts to open a crate with a toothpick would have a higher CL than the same character attempting to open it with a crowbar.

Now that you understand Dice Pools, Actions and Challenge Level, we can look at an example roll in detail.

\textit{Example}

A character attempts to pick a lock – the Narrator assigns a CL of 3 (a specialized task) and determines that Sleight of Hand is the appropriate skill. The player is happy with this and doesn't offer an alternate approach.

The player forms their dice pool by adding their Base Three (3D6) to their Sleight of Hand skill. The player has two points in Sleight of hand (2D6). Finally, the player notes that Sleight of Hand uses Reflexes as its Core Attribute, and adds one point in Reflexes (1D6).

The player rolls their Dice Pool (6D6) against the CL. The players' roll is successful if the number of wins is equivalent to or greater than the CL assigned to the task.

They score four successes, and the narrator describes their success, the lock is open!

\section{Skill Speciality} \label{sec:skill_speciality}

Once someone has specialized in a given task enough, they can perform its simpler and more routine functions with ease. Once your dice pool for a given skill check is equal to or greater than \textbf{EIGHT} you are considered a specialist in that skill. If you are a specialist for a skill when asked to make a check for that skill with a Challenge Level of \textbf{THREE} or less, you automatically pass if all the following conditions are true:

\begin{itemize}
    \item You are not under duress (No time pressure, no outside influence or aggression such as combat.)
    \item The check is not a contested check.
    \item The check is not a reflexive action.
\end{itemize}

NCE: CHSH - Active/Passive Notes