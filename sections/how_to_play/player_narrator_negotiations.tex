Cogent is a flexible system with purposeful provision for interpretation and adaptability. We’re not going to spell every single thing out for you (like specifying each and every circumstance where a specific skill can be used to assist another skill, or spelling out every possible vocation you’re allowed to pick.) If we did this we feel it would make the game system restrictive and complex, which would result in a worse game by the end.

You’re smart and we trust you to see ways in which certain aspects of Cogent can benefit your character, in things such as skill interaction, assists and combat modifiers.

However, because every person is an individual, they’ll very likely see different ways certain things in the game can be interpreted and thereby how they feel it should affect the gameplay. This is wonderful and we encourage every player and narrator to explore their creative ideas as it enriches and enhances the gaming experience.

In these situations, it becomes important to familiarise yourself with the process of friendly player/narrator negotiation. If you as a player honestly feel that your character should receive a bonus, such as an additional dice to roll with, because of some kind of in-game circumstance, you should feel free to suggest this to the narrator. Narrators shouldn’t feel defensive if a player makes such suggestions – this is a great thing and shows how engaged your players are in your game. Try to give fair consideration to every suggestion or request a player puts forward. We’re not saying you have to agree to everything, only to give honest consideration to such suggestions, and explain why you disagree if you do.

Players, if the narrator disagrees with you, once you have presented your argument, it is then your responsibility to accept the Narrator's decision and don’t argue further! That’s when the role-play experience gets much more frustrating, so if you feel the Narrator is giving you “lemons”, just try and make “lemonade”.

\section{The Narrator shouldn't hold your hand} \label{sec:narrator_hold_hand}

You should be creative to survive. Even in the most hopeless situations, clever thinking can achieve a great deal. Don’t just rely on your character statistics to win conflicts in your game. Remember you are creating a story with your friends, so approach everything in the game in a narrative way. Don’t just say “I attack”, describe how you are attacking, and think outside the box every once in a while, and a good Narrator will reward clever role-playing.

\section{Roleplay can progress the game} \label{sec:roleplay_progress_game}

Roleplay can progress the game as much as combat. An entire roleplay session can be very rewarding, even if a single combat dice aren’t rolled. The main element in a role-playing game is role-playing, not combat, so don’t expect to fight every time you play. There are plenty of other challenges and dice rolls that will need to be done outside of combat and never think that combat is the only way to get out of sticky situations. Remember the key is to do what your character would do, so if your character would resolve most conflicts through combat, then go right ahead, but there are many other character types to play other than battle-focused adventurers, and they can be just as enjoyable.