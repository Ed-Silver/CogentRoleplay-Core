Core Skills, Proficiencies and Vocations can all be used to perform assist rolls. A player can make an assist roll to (potentially) gain additional dice to aid their primary action/objective. Assist rolls are made before the action in which the player wishes to receive the assist. Assists may be applied to any logically applicable roll - and the player must first ask the Narrator if the assist roll is appropriate.

If the player is in a position where a skill they possess would logically assist or grant an advantage to another skill or combat roll (whether their own or another player’s) the Narrator can assign a Challenge Level (\textbf{CL}) based on the difficulty of the assist.

\underline{The \textbf{CL} for assists starts at three and should never be lower than that.} However, it is up to the narrator to decide if it should be higher than three. In general, most assist roles, unless extremely unlikely to aid the current situation, will be made at \textbf{CL 3}.

If the assist check was successful, additional wins are added to the result of the skill or combat roll the player has assisted - equivalent to the number of wins achieved OVER the Assist \textbf{CL} in the assist check.

\textit{Example:}

\begin{displayquote}
    3 wins against an assist CL of 3 does nothing to assist the following action, but 4 wins will grant +1 win to the result of the assisted action's roll.

    On the other hand, if a player achieves only 1 win in an assist roll of CL 3, the player will then face a penalty of -2, to be deducted from the result of the assisted action's roll. This means that any assist can backfire, and as such carries an element of risk.
\end{displayquote}

Each player can only attempt 1 assist each round. This means that if a player has 2 skills that could logically assist another skill check or combat roll they must choose between those 2 skills.

The number of wins gained or losses subtracted from an assist can never be greater than the challenge level of the assist. For instance, you can only ever receive an additional 3 wins or losses to a skill or attack that you’re trying to assist if the assist check was \textbf{CL 3}. Likewise, an assist check with a \textbf{CL 4} will add a cap at max 4 wins or losses depending on the outcome of the assist check.

Every skill, vocation, vocational skill and combat skill can be used for assists, provided it is logical and the narrator approves it.

\textit{Example Scenario:}

\begin{displayquote}
    You wish to make an acrobatics check (\textbf{CL 4}) but the CL is one point less than the total number of dice in your dice pool (\textbf{5D6}), a hard check to make - however you also have two points in the vocation 'martial artist'. The narrator approves of your quick thinking and allows you to roll to perform an assist to your acrobatics skill check.

    Your total Martial Artist vocation dice pool is 7D6 and the narrator sets the assist at \textbf{CL 3}. You roll five wins total, meaning you will now receive +2 wins toward the result of your following acrobatics check.

    The acrobatics check is \textbf{CL 4} and as previously mentioned your dice pool is 5D6. You roll four wins which would mean a loss if not for your assist, providing +2 wins making a new total six, meaning through using an assist you have succeeded a difficult Challenge Level!
\end{displayquote}

Within Cogent Role-play, we don’t outline every applicable situation where a given skill can be used. It’s up to the players and the Narrator to negotiate how these skills can be logically applied to different circumstances.

Having said this, if you are acting as the Narrator you will find that players will try for some very thin connections between a given skill they possess and the situation they are in to attempt a skill check or a possible assist. The players must feel free to express their logic to the Narrator and the Narrator should listen to all requests fairly – cooperative negotiation is something that the players and narrator should engage in as it helps the narrative reach new creative levels – but in the end, it will be the narrator’s call and the players must accept what the Narrator decides.

\section{Helping Others} \label{sec:helping_others}

Thus far, assists have only been described when used by the player to assist themselves with another skill. However, players are free to use assists to help (or unintentionally hinder) others as well. In this instance, it is up to the narrator to determine if it is possible for someone else to assist, and if so, how many people can attempt to assist. This limitation is usually drawn from what would make narrative sense. It would be nigh impossible for someone to assist in a computer hacking check if the person hacking was using the only computer available. However, if a second or more computers were on hand, it would make logical sense that others could attempt to aid the player in rolling the hacking skill check.

In most situations, when multiple players are attempting to complete the same skill check to achieve the same task (eg: investigating a crime scene) the narrator will call on them to decide who is the primary actor in this situation and who is assisting, rather than each player making a separate skill check. Usually, the exception to this is when multiple players are making physical skill checks to perform a highly individual activity at the same time. (EG: Climbing a wall). While they are all attempting the same type of skill check, they are not attempting the same task, as each is rolling to climb up the wall as an individual, not a group. They succeed and fail alone.

\section{Single Person Assist} \label{sec:single_person_assist}

When assisting a single player, first you need to determine if they want your help at all. We recommend playing this scene out in character, and offering your assistance, most of the time they will be open to the idea, however, occasionally an overconfident or strong-willed character may refuse. At this point, it is up to you to decide if you will heed their wishes and help or refrain from helping. You are free to ignore what the other player wants and attempt to help regardless IF the narrator allows it. Some tasks, such as looking for items, investigating, and checking someone’s vital signs can all be done by multiple people at the same time, however other tasks are impossible to assist with without consent. Remember, your assist roll can help OR hinder, so beware that these kinds of forced assists can cause friction between characters. This can create amazing narrative tension and be very true to the character, but try not to do it if it is irritating the other players at the table.

If you assist in this way, it functions identically to assists for a single character. You roll a skill assist and the active person making the skill check adds wins or subtracts losses from their roll based on your assist roll.

\section{Group Assists} \label{sec:group_assists}

As previously mentioned, as many people can assist with an activity as the narrator deems sensible or appropriate. This could be none, one or fifty. It is the narrator’s role to make it clear how many people can helpfully assist. While in general, there is always a chance that an attempt to aid a situation will backfire, when greater numbers of people pool their efforts, the average begins to swing in the favour of aiding the situation. Assign a player who is making the skill check, then, assign another player who is assisting and what skill they are assisting with. Each additional player adds a single dice to the dice pool of the player making the assist roll.

\textit{Example:}

\begin{displayquote}
    Gary is trying to recall if there are any back-alley weapon traders in their local area. He goes to make a General Knowledge skill check, when Lucy offers to aid with her Vocation (Street Thug). The narrator deems it more than suitable to assist in this kind of check, and offers that anyone present that has spent significant time in the area has a chance of adding more information. The two other players, Bill and Ted, both say they wish to assist.

    Mechanically, Gary is making a General Knowledge skill check. Lucy is making a Vocation (Street Thug) assist. She has a Dice Pool of 6D6, but with Bill and Ted assisting, her dice pool is increased to 8D6 for this roll (One dice added per additional player assisting). Lucy rolls and scores 6 wins, this beats the assist CL3 and adds a total of 3 wins to Gary's General Knowledge skill check. Seems like Lucy's time on the street paid off.
\end{displayquote}

\textbf{Note - characters, action economy and trust:} In reality, in a functioning party, most of the time people will trust each other to complete their assigned roles. If you are directing your parties’ horse and cart, and another player is composing a song in the back of the cart, you will be relying on the characters you have entrusted to act as lookouts to do their job and have no reason to suspect they cannot. Often times, the narrator will disallow assists based on the fact that your characters are already actively doing something else. In this situation, assisting a perception check to look out for danger may be limited to the number of players not actively engaged in other activities.
