\section{Introduction} \label{sec:introduction}

The first thing that is needed prior to playing is a group of players. Once you have gathered some friends and determined who is going to be the Narrator, the Narrator will begin planning the story and the players will build their characters. This will be covered in more detail in Chapter 2 (Character Creation) and Chapter 5 (Narration). But before you can launch into your adventure, it is important to understand the core mechanics of Cogent Roleplay. While Cogent is a ‘Roleplay First’ system, the mechanics facilitate the stories that you tell, and as such have been designed to be easy to understand as well as robust and universal.

Throughout this ruleset, you will find rules presented for \textbf{Fastplay}. This refers to most games of Cogent Roleplay where a group picks up and plays for one or a few sessions. Modifications to the standard game rules can be found under \textbf{Campaign Play} in Chapter 5: Narration. These modifications mainly cover character creation stat distribution and awarding character advancements during play. If a listed rule references how it operates in \textbf{Fastplay} and is not modified in the rules or by the narrator for \textbf{Campaign Play}, assume the \textbf{Fastplay} method still applies.

\section{Dice Pools} \label{sec:dice_pools}

Dice rolls in Cogent Roleplay fall into two categories, unopposed Actions and opposed Conflicts, both requiring the player to roll a group of dice known as a Dice Pool.

Cogent Roleplay uses traditional six-sided dice for its gameplay, known colloquially as \textbf{D6}’s. (D indicates “Dice” and the 6 refers to the number of sides.) When making any roll in Cogent, you gather a group of \textbf{D6}s together and roll them as a group.

Your dice pool is determined by the stats on your character sheet and is made up of three parts. The exact type of roll you are required to make will be called by the narrator, and will always be linked to a skill on your character sheet. (EG: General Knowledge.)

Let’s look at forming the Dice Pool you will use in every roll you are required to make while playing a game.

\textbf{Step 1: Add Base Three:} Your dice pool always starts with 3 dice. This represents the capacity for an untrained person to always have some ability to attempt any given task. These core three dice are included in every single check and are often referred to simply as your ‘base three’.

\textbf{Step 2: Add Core Attribute:} Core Attributes represent the natural talents and capacities of your character. Each skill you are using is governed by a Core Attribute, you also assign points to these attributes during character creation. Any points assigned to a relevant core attribute are also added to your dice pool. The skills on your character sheet are grouped under their relevant attribute. (For example: Infiltration has “Intelligence” as its Core Attribute. If you are making an Infiltration check, you would add any points of Intelligence you have marked on your character sheet to the dice pool.)

\textbf{Step 3: Add Skill Points}: Skill points represent the specific knowledge and practical skills of your character. When being asked to make a check to attempt a task, one of your 15 core skills or any of your vocational skills will be used. When creating characters, you assign points to these skills as you see fit. These points correspond to dice that you add to dice pools. (For example: If you had an ‘Infiltration’ skill with 2 points assigned to it, you would add 2 dice to your dice pool in this step).

\textbf{Step 4: Apply Modifiers:} Finally, occasionally during games your character will have modifiers to their dice pool, whether via injuries, unique characteristics or situational penalties. The modifier will be a number, and that number represents how many dice you add or remove from your dice pool.

Note: It is entirely possible for a character to have negative points in a skill or attribute, if this is the case, instead of adding dice to the pool, subtract them at the relevant step.

\textit{Example:}

\begin{displayquote}
    The Narrator asks Rob to make an "Athletics" check. Rob's character has 2 points in Athletics, and 1 point in Strength.  Since the Narrator didn't declare there were any Modifiers, Rob rolls \textbf{6D6} (Base Three + Strength + Athletics $\pm$ Modifiers).
\end{displayquote}

Now that you understand how to form a dice pool, it is time to move on to how those dice pools are utilized and what they are rolled against.