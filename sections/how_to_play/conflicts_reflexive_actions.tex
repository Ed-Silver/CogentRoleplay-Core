Situations may arise in which a player must respond reflexively to a threat, which may have the potential to harm or injure them. In these situations, the Narrator will tell the player to respond in a specific way, which will vary based on the threat.

Generally, reflexive actions can be used in places where other systems might ask for a ‘saving throw’ but they can also be used where players have an obvious physical cost associated with failing to achieve something. For example, navigating hostile environments.

In the context of these checks, after rolling to meet the \textbf{CL} required, if you failed, ascribe injury to a character based on the difference between the number of \textbf{wins} scored on your roll and the number of \textbf{wins} required to meet the \textbf{CL} of the action. (within the context of the level of danger the task entails.)

\textit{Example:}

\begin{center}
    \begin{xltabular}{\textwidth}{|X|X|X|} 
        \hline 
        \textbf{Challenge} & \textbf{Roll} & \textbf{Result} \\ 
        \hline
        An “Acrobatics” check while falling from a dangerous height. & 6D6 vs CL 4 and 3 wins & The character missed the CL by 1 and receives a level 1 injury (see \textbf{Injury} under \textbf{Combat}) \\
        A “Survival” check to find food in the wilderness & 7D6 vs CL3 and 4 wins & The character successfully finds food \\
        An “Endurance” check to resist the effects of an ingested poison & 7D6 vs CL 6 and 3 wins & The character missed the CL by 3 and receives a level three injury (see \textbf{Injury} under \textbf{Combat}) \\
        \hline
    \end{xltabular}
\end{center}

\section{Conflicts} \label{sec:conflicts}

A conflict occurs when an action is taken against another character that can actively oppose it. Conflicts may arise in various forms, such as combat, arm wrestling, seduction, or sneaking up on someone. Both parties involved will be required to make a roll, using their dice pools based on relevant narrator-assigned skills.

Within conflicts where all characters are considered aware and active, rolls that are tied/equal are simply re-rolled.

Examples of conflict that involves multiple active opponents:
\begin{itemize}
    \item An Arm wrestle
    \item Combat
    \item A foot race
    \item Bartering with a shopkeeper
\end{itemize}

In some situations, rolls that would usually be conflicts may instead be assigned a \textbf{CL} by the narrator. This is usually only necessary when the character that would oppose the conflict is completely powerless to do so by normal means. For example, sneaking up on someone having a meal would be a conflict, but if they were sleeping and unaware the narrator could instead assign a \textbf{CL} to reflect that. As a general guide, the \textbf{CL} of such a check should be lower than half that character's dice pool. This method should only be used for players acting on \textbf{NPCs}, and never the other way around.

Most Conflicts are simple and require each character to make one roll. When conflicts take several rolls to resolve (such as in combat) the conflict is conducted in a Round by Round format.

SEE CHEAT SHEET REFERENCE: CHSH.2 – Active/Passive Notes