%[Error!] missing start of sentence! 
that can be used when employing ballistic weapons. WARNING, these rules are called hardcore for a reason. It is VERY easy to die when using these rules, just like in real life when dealing with ballistic weapons. If these rules are not used when employing ballistic weapons the players will have an active chance to defend against ballistic attacks, such as dodging. While this isn’t realistic it does convey a superhuman level of combative skill and if that is the campaign you want to play, the regular rule system will work perfectly. Indeed both these rule systems can be used, the regular people having to follow the hardcore rules while the exceptionally skilled/superhuman/scientifically advanced people get to follow the regular combat rules.

The key difference in the hardcore rules is that ranged combat rolls are done against static \textbf{Challenge Levels} determined by the distance and size of the target. Their total combat roll is calculated exactly the same, except that they are not trying to beat another player’s roll. Every win over the CL stands as a level of victory the player achieves against their opponent. Those attacked by weapons that use the hardcore rules cannot make any defence roll at all. Ballistic weapons (or other similar weapons like laser guns) work differently to, say, bows, because one cannot actively defend against a fired bullet. They can prepare before the gun is fired such as ducking behind the cover or wearing armour, but once the gun is fired and its aim was true, you’ll get hit.

\begin{tabular}{|m{.14\textwidth}|m{.14\textwidth} m{.14\textwidth} m{.14\textwidth} m{.14\textwidth} m{.14\textwidth}|}
    \hline
    \textbf{Range} & \textbf{Stationary} & \textbf{Stationary up to 50\% cover} & \textbf{Stationary up to 90\% cover} & \textbf{Moving Target (Human speed)} & \textbf{Moving Target (Car speed)} \\
    \hline
    Adjacent & \multicolumn{1}{c}{1} & \multicolumn{4}{c|}{Any target under these conditions is not adjacent} \\
    \raggedright Close Range \newline 5 to 10 meters & \multicolumn{1}{c}{2} & \multicolumn{1}{c}{4} & \multicolumn{1}{c}{6} & \multicolumn{1}{c}{4} & \multicolumn{1}{c|}{8} \\
    \raggedright Medium Range \newline 10 to 50 meters & \multicolumn{1}{c}{4} & \multicolumn{1}{c}{6} & \multicolumn{1}{c}{8} & \multicolumn{1}{c}{6} & \multicolumn{1}{c|}{8} \\
    \raggedright Long Range \newline 50 to 100 meters & \multicolumn{1}{c}{6} & \multicolumn{1}{c}{10} & \multicolumn{1}{c}{14} & \multicolumn{1}{c}{8} & \multicolumn{1}{c|}{10} \\
    \raggedright Sniper Range~* \newline 100~m to 1~km & \multicolumn{1}{c}{6} & \multicolumn{1}{c}{10} & \multicolumn{1}{c}{14} & \multicolumn{1}{c}{8} & \multicolumn{1}{c|}{10} \\
    \hline
\end{tabular}

\textit{*~Sniper range can only be attempted if the weapon has a scope.}

These CL’s are doubled if one of the following conditions are not met:

\begin{enumerate}
    \item You don’t have at least one proficiency in sniper.
    \item You are not kneeling or lying down.
\end{enumerate}

It's also important to note:

\begin{itemize}
    \item Walking is considered stationary
    \item If the target is moving towards or away from you, it is considered stationary
    \item If the target is moving and they have cover the CLs are added together
\end{itemize}