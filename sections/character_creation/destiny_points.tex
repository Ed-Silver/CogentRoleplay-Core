In Cogent we want you to create a character that feels real and has motives, desires, goals, flaws quirks and opinions. We feel that the more in-depth your character is, the more you will be drawn into the game you are playing. As a type of reward system to encourage this, Cogent Roleplay implements what we call Destiny points, which should be awarded to players for exceptional role-playing.

A Destiny Point represents the character’s influence on fate, their place in the universe and how their will may affect it. It can be spent in either one of two ways:

\subsection{Before any roll of a dice pool} \label{subsec:before_any_roll_of_a_dice_pool}

A character may spend a Destiny Point before any d6-based roll, usually a particularly important one if they’re using their points wisely, which will give them a special destiny advantage in the roll. Instead of the usual 4-6 on a d6 being considered a win, each dice that rolls a 3 or higher will be considered a win, meaning 3-6. This increased the chances of rolling winning dice considerably but does not guarantee it, there is still a chance of failure.

\subsection{After any roll of a dice pool} \label{subsec:after_any_roll_of_a_dice_pool}

A character may spend a Destiny Point after any d6-based roll to add a single win in addition to what they achieved. They may do this multiple times on any roll or separate rolls, for as many Destiny Points as they have.

\section{Awarding Destiny Points}

In \textbf{Fast Play} every character starts with 3 Destiny Points at character creation (unless the Narrator says otherwise). This is also the maximum number of destiny points a player can have. During \textbf{Fastplay} games award one destiny point roughly every three hours of play as a baseline. During \textbf{Campaign Play} destiny points are awarded at different rates depending on the tier of play. The narrator may also award players additional destiny points throughout the game for exceptional role-playing. Examples range from the player doing something that works towards their character’s goals, when they achieve something significant, or they do something, particularly in line with their character’s personality even if it was at the detriment of their wealth or progress.