Vocations represent the path your character has walked up until the beginning of the game. They may continue to travel down that path, or may change going forward, but these vocations reflect the life they led and the career they chose.

Vocations in cogent roleplay are chosen in much the same way as Core Skills, and use skill points taken from the same pool during character creation. However, unlike Core Skills, Vocation selection should be discussed with the narrator before being finalised.

When selecting a Vocation or Vocations, a player must choose a career. These should be fairly specific, but must always represent a Job as opposed to a task or activity. For example, a player could choose to be a Chef, but not “cooking”. The player has the freedom to pick almost any vocation, but can’t pick all-encompassing metagame vocations like, ‘God’, or ‘master of everything’. It has to be a logical and pre-established Vocation fitting within the game world the Narrator has prepared

Selecting at least one Vocation is \textbf{Mandatory} and as such you get \textbf{1 free skill point} to assign to your first vocation. If you wish for your character to be unemployed, life goals or ideals are also suitable choices – “Drifter” or “Murderer” are perfectly fine selections for a Vocation. They are meant to reflect the path your character walks and the skills and talents they have learned within it. Essentially vocations can be seen as the character class component of Cogent Roleplay. A vocation can be many things such as classic fantasy character classes like knight, druid, wizard, ninja, soldier or more mundane vocations like blacksmith or baker. Much like core skills, Vocations can be applied to anything. Even a Lion would have a vocation such as “Plains Predator.”

Once you have chosen one or more Vocations, the player and narrator should agree on an appropriate Core Attribute to apply to each Vocation. This attribute is the one added to dice pools for your vocation.

Vocations are unique among Cogent skills. They cannot be used to make Skill Checks. A Vocation is essentially an ‘assist’ skill, meaning in most situations, it can only be used to make assist rolls to other skill checks or combat rolls, and only if the vocation has a logical and obvious association with the skill or combat roll you are trying to assist (and is approved by the Narrator). (See Chapter Blah, Assists)

\begin{figure}[H]
    \includegraphics[width=8cm]{images/placeholder}
    \centering
    \caption{Image Placeholder.}
\end{figure}
%[alt_text](images/image4.png "image_tooltip")

With a Vocation chosen, you may now select up to four skills to write underneath the vocation.

These may fall under two categories. \textbf{Vocational skills} and \textbf{Combat Skills}. These skills can be determined by the player but must be agreed upon by the Narrator. In \textbf{Campaign Play}, you do not need to select all of these skills at character creation, with empty slots able to be filled later as you earn skill points. Both Vocational Skills and Combat Skills must be checked and approved by the narrator, and they must make logical sense as talents learned in pursuit of the chosen vocation they nest under. For example, you could not choose the combat proficiency “Heavy Ballistic Weapons” if your chosen vocation was “Nurse”. However, you could take “Small weapons” to reflect deft hands with needles or scalpels.

You can assign up to four points to any Vocation. You can also assign up to four points to any Vocational Skills or Combat Skills. However:

\textbf{You may never assign more points to a Vocational Skill or Combat Skill than you have assigned to its parent Vocation.}

(EG: A player with Vocation: Chef and two points assigned to it may not assign a third point to its Vocational Skill: Cooking, until they have first assigned a third skill point into Vocation: Chef)

Finally, Vocations, Vocational Skills and Combat skills may NEVER be assigned a negative skill point at character creation.

\section{Vocational Skills} \label{sec:vocational_skills}

These skills showcase the unique talents and specialized skills you have learned while following the path of your chosen vocation. They are always allowed to be used for assist rolls, but may only be used for direct skill checks if that check is not better represented by a Core Skill. Using the example of Vocation: Hunter, here are some example Vocational Skills:

\begin{itemize}
    \item \textbf{Tracking:} When a Hunter comes across tracks, they will be able to (in order of difficulty); follow the tracks, know their target destination, identify the species, or identify their health
    \item \textbf{Wild Craft:} This skill grants the Hunter the ability to use materials in the wild to build traps, shelter, wooden spears, and other tools used to survive in the wilderness long term.
\end{itemize}

These skills can be extremely specific, and while their function can be linear, it is designed to allow the players' skills to open up opportunities that would be impossible for untrained characters. For example, without a Vocation such as “Computer Specialist” and the Vocational Skill “Hacking,” it would be impossible for a player to know how to perform the task.

Much like the Vocations, they are nested in, players and narrators assign the Core Attribute for EACH Vocational Skill, this means it is possible to have a Vocational Skill with a Core Attribute that differs from its parent Vocation.

For example: Vocation: Warrior (Strength) may have Vocational Skill: Battle Tactics (Intelligence)

\section{Combat Skills} \label{sec:combat_skills}

Combat skills are selected as part of the Vocations they were learned in. They differ from other skills in two extremely important ways.

They must be selected from a predetermined list of options that correlate to a type of weapon modifier in cogent. These are:

\subsection{Melee} \label{subsec:melee}

\begin{itemize}
    \item \textbf{Unarmed:} Martial Arts, and “Hand” weapons like knuckle dusters.
    \item \textbf{Small Weapons:} Knives, Daggers, Saps, and other easily concealable weapons.
    \item \textbf{Medium Weapons:} Backup/Personal protection weapons like shortswords, or clubs
    \item \textbf{Large Weapons:} Secondary battlefield weapons like Maces, Longswords, and Flails
    \item \textbf{Reach Weapons:} Primary battlefield weapons like Polearms
\end{itemize}

\subsection{Ranged} \label{subsec:ranged}

\begin{itemize}
    \item \textbf{Short Stringed:} Small ranged weapons including blowguns, throwing weapons, and hand crossbows.
    \item \textbf{Long Stringed:} Standard medieval battlefield ranged weapons, like bows and crossbows
    \item \textbf{Hand Ballistic:} Pistols, and Sub-Machine Guns
    \item \textbf{Shoulder Stocked Ballistic:} Standard battlefield firearms like Muskets, and Assault Rifles
    \item \textbf{Heavy Ballistics:} Machine Guns, Cannons, and other Siege weapons
\end{itemize}

The second important difference is that Combat Proficiencies are the only skills that can be used in Combat which are covered in full detail in Chapter 3, Combat. Combat skills also accommodate things like supernatural abilities (magic or superpowers). If there is to be magic within a campaign setting, a player may need to specify a sub-type of magic as a Vocational Skill or Combat Skill by spending points on it to be able to use it in-game (see CH4, “Narration” and CH7, “Example Magic System”).

If your game is using a Magic system or similar (IE: Superheroes), consult with the narrator to determine applicable Combat Skills as they will not appear on the lists above. Further guides on establishing magic settings and how to balance them as Combat Skills can be found in Chapter 5: Narration. An example magic system can be found in Chapter 7: Supplementary Content.