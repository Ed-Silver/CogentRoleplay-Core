Playing as an ‘unstoppable fighter’ can be less satisfying than playing an average fighter who struggles to beat a strong challenge. The greater the challenge a player overcomes, the more satisfaction the achievement. This concept also applies to characteristics. Often, disabling characteristics can create a richer and more entertaining character.

Taking on such a disabling characteristic does restrict the way in which you’ll play your character in-game, that’s why they are called disabling characteristics. It will make your role-play more challenging and thereby more satisfying. However, because such characteristics are restrictive, it’s only fair that the player who chooses to have one for their character receives an equivalent bonus. This bonus is also in place to encourage players to take on such characteristics who wouldn’t do so normally.

If you choose a Disabling Characteristic from the table below or create one that the Narrator approves, you will receive one additional Skill point to spend on character creation.

Alternatively, you may roll a twenty-sided dice, the resulting number corresponding to the table below, and take the Disabling Characteristic associated. This cannot be taken back. If you choose to roll for a Disabling Characteristic you must keep it! It will forever be a part of that character. In doing this you will receive two additional skill points to spend.

A player may apply (or roll for) as many disabling characteristics as they want unless the Narrator stops them past a certain point. However, you will only receive bonus skill points for the first TWO disabling characteristics you take (whether selected, rolled for, or in combination).

You don’t need to take a characteristic listed on the table provided, you’re free to make up whatever disabling characteristic you can think of, and your Narrator will decide if the characteristic you have chosen is disabling enough to receive the standard bonus given in the rules.

Disabling Characteristics should always be chosen or rolled before any points are assigned to Skills, otherwise often you may find that a disabling characteristic might utterly disqualify a skill you have chosen. For example, it’s obviously hard to use the skill deception if you have the Disabling Characteristic, ‘Can’t Lie.’

Narrators should also be aware that it is possible for the player characters to overcome many disabling characteristics through the narrative of the story. A character whose life was saved by a species which they have an extreme prejudice against might rightly have their prejudice soften or even go away. These evolutions of character should feel earned and justified, they are important moments and should be treated as such.

SEE CHEAT SHEET REFERENCE: CHSH – Disabling Characteristics

Following is a list of 20 disabling characteristics a player may roll for, including detail:

Many of these characteristics are open and vague, leaving it up to the Narrator to define specifics, provide definitions, examples and penalties.

Some of the following disabling characteristics cause an increased cost of certain skills during character creation. These are marked with an asterisk. ( * ) If an attribute is marked, all skills that use it as their core attribute suffer the cost increase. This further increases the importance of selecting or rolling for disabling characteristics prior to assigning skill points during character creation.

\begin{enumerate}
    \item \textbf{Missing a bodily extremity/limb}
    
    The obvious example is that the character could have lost their hand or foot, but this may also include another extremity such as a nose or a reproductive organ.

    \item \textbf{Feeble} (-1 to all \textbf{Strength}* based skills and \textbf{Strength} Checks)
    
    This may mean the character is malnourished, genetically weak, or maybe the result of an injury. The character is far weaker than a normal person.

    \item \textbf{Heavy} (-1 to all \textbf{Reflex}* based skills and \textbf{Reflex} Checks)
    
    The character may be big-boned, have lots of muscle, or be excessively overweight.

    \item \textbf{Dim-witted} (-1 to all \textbf{Intelligence}* based skills and \textbf{Intelligence} checks)
    
    A result of a bad head injury, or the character may simply be stupid.

    \item \textbf{Phobia}
    
    This is a debilitating fear that the character possesses that is to be selected by the narrator and can range from heights to spiders or little children.

    \item \textbf{Extreme Prejudice}
    
    \ldots against a species, culture, sexual orientation, political affiliation or religion. The Narrator is free to be creative, for example, making the character want to kill every person that supports a specific sports team or hold irrational hatred for those who wear anything with ribbons.

    \item \textbf{Compulsive Liar}
    
    This characteristic needs to be played in the right way – the player shouldn’t lie about everything to the point that it’s impossible to hold a conversation with them. Generally, it would be about anything concerning their achievements or anything that would give them an advantage no matter how small, like claiming the gold coin a character found on the ground, or inventing feats of heroism to impress locals.

    \item \textbf{Kleptomaniac}
    
    Kleptomania is not necessarily the desire to steal but rather a compulsive need to have moszt things they see for reasons other than personal use or financial gain. Many of the items a kleptomaniac steals are useless to them and the item’s value does not weigh in on their compulsion at all, thus most items a kleptomaniac steals are worth very little.

    \item \textbf{Paranoia}
    
    The extreme belief that every person they meet will eventually betray them, that the government is run by an evil cult or perhaps that stepping on a crack in the pavement will cause their death.

    \item \textbf{Over-emotional} (specific emotion)
    
    An over-emotional character has a specific emotion that causes significant issues for them. It could be an inability to control their anger that senses them into senseless rages, or perhaps an incapacitating and incontrollable grief that is triggered when being reminded of past trauma. Extreme joy can be equally problematic, a character who simply cannot contain their constant joy and enthusiasm would make fast enemies at an important funeral.

    \item \textbf{Very Forgetful}
    
    This does not mean forgetting everything to the point that the character is an unintelligible drooling oaf. This characteristic can manifest in such ways as a character always forgetting where they are and getting lost all the time, forgetting plans, directions or instructions. They might never be able to remember the names of people, towns or countries, or they may constantly forget where they put their items.

    \item \textbf{Short Tempered}
    
    A character with a short temper will react out of proportion to the norm when angry. The subject of the characters’ short temper may be broad (they are generally angry – the ‘Ebenezer Scrooge’ type), or specific (small animals make them furious!)

    \item \textbf{Overconfident} (cannot spend points on an applicable skill during character creation)
    
    You are extremely overconfident in a specific skill, such as stealth, and will feel that you’ll always succeed in any activity using it. You will readily resort to any method that can employ the said skill and volunteer to use it whenever the opportunity arises. That unfounded confidence has caused no effort to be spent actually learning or training in the skill.

    \item \textbf{Incapable of Lying} (specific parameters required)
    
    This is not a blanket restriction, the character will be able to lie, but not in circumstances the narrator specifies – such as not being able to lie to a certain gender. It might also mean a character can’t speak any ‘untruths’ but are still able to mislead people (like the Aes Sedai in “the Wheel of Time” novels.) It may also mean the character has a very prominent physical tell whenever they lie such as sneezing or farting.

    \item \textbf{Addiction} 
    
    This is not simply a preference for a certain thing, it is a disabling addiction, something that the character has withdrawals from if the addiction isn’t satisfied often. The addiction itself can include things like sex, drugs, specific foods, or even murder.

    \item \textbf{Delusion}
    
    Your character truly believes something to be real beyond the scope of simple faith. They see and hear and feel that belief in their own reality. Examples: having an imaginary friend, hearing voices, or honestly believing that pot plants are secretly an invading alien race.

    \item \textbf{Imperceptive} (\textbf{-3D6} to environmental \textbf{Perception} rolls)
    
    Your character is withdrawn, paying little heed to the world around them as they move through it. They could be a daydreamer or a shut-in, but either way, they are the sort of person most likely to walk out onto a road without looking, oblivious to the danger.

    \item \textbf{Gullible} (\textbf{-3D6} to \textbf{Perception} rolls \textit{vs} \textbf{Deception})
    
    Liable to believe what is told to them, a gullible character is terrible at detecting lies and motives and generally believes in good faith that what they are told is true. While they can tell particularly outlandish lies, they’re at a significant disadvantage detecting good ones.

    \item \textbf{Impressionable} (\textbf{-3D6} to \textbf{Perception} rolls \textit{vs} \textbf{Persuation})
    
    Easily convinced by friends and trusted allies, even by regular folk. Impressionable characters are easily convinced to take particular paths to action and easily convinced to lend a helping hand to those who ask for it.

    \item \textbf{People Pleaser} (Overly agreeable / Yes person)
    
    Your character will do whatever they can to make other people happy, often at the expense of their own happiness or goals.
\end{enumerate}
