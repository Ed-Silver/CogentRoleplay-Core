The commerce point system is optional and the Narrator can choose to use it or resort to a traditional numeric method (keeping track of the number of dollars or gold coins a player has earned or spent.) The Commerce point system provides an easier way to govern the commerce of your games without the need to constantly calculate purchases and total currency amounts.

Below is an item value system that ranges from 0 to 8 that is to be used by the narrator to help calculate the value of any given item in the game.

SEE CHEAT SHEET REFERENCE: CHSH – Commerce Scale

\begin{center}
    \begin{tabularx}{\textwidth}{|c|X|X|} 
     \hline
       & \textbf{Commerce Points and Level of Wealth} & \textbf{Item Value} \\ 
     \hline
     \textbf{0} &   & A simple meal, second-hand clothes, a can of beer, a magazine, admission to a public venue such as a pool or gym. \\ 
     \textbf{1} & Pauper & Basic food for the day, a night at an inn, a new item of clothing, a sturdy rope, a bottle of rum. \\
     \textbf{2} & Commoner, Paycheque to Paycheque Worker & An extravagant night out, an adequate weapon, common livestock, camping gear, simple transport (old car, mule and cart.) \\
     \textbf{3} & Tradesperson, craftsman & A well-crafted weapon, a decent horse/car/vehicle, average armour, a small building/house/flat. \\
     \textbf{4} & Accomplished Career, Small Business Owner & A very fancy horse/vehicle, a nice home, extravagant weapon, full plate armour, bodyguard. \\
     \textbf{5} & Manager, Politician, Successful Merchant, Minor Nobility & A mansion, luxury transport, estate, a personal guard or security force, established business. \\
     \textbf{6} & CEO, Nobility & A castle/skyscraper, enterprise level company, large army, significant military asset, a town. \\
     \textbf{7} & Baron, Magnate/Tycoon & Metropolis, armada, small country. \\
     \textbf{8} & King/Queen, Head of Megacorporation & Kingdom, nation, established planet (sci-fi.) \\
     \textbf{9} & Kingdom, nation, established planet (sci-fi.) &  \\
     \hline
    \end{tabularx}
\end{center}

Characters and NPCs have a “Commerce Level” that defines how wealthy they are in a given setting. This level is set during character creation in collaboration between the player and narrator.

Players can purchase any item that has a value score equivalent to, or less than, their total commerce points. Purchasing an item/service with a value equivalent to your current level of commerce points will subtract/cost 2 commerce points from your total pool.

Purchasing an item/service with a value equivalent to 1 point less than your current level of commerce points will subtract/cost 1 commerce point from your total pool.

Purchasing an item/service with a value equivalent to 2 points under or less than your current level of commerce points will not remove any points from your total pool – unless purchased in large quantities or regularly.

One way to think about it might be like this:

Purchasing an item with a value equal to my level of wealth, will cost a lot of my total finances (such as someone with a high income [4 commerce points] buying a nice home [lvl 4 purchase, costing two commerce points], reducing their financial status to ‘commoner’ [level 2] after the purchase is made.)

This ratio of cost for the specified item values is present because commerce points go up exponentially in value. If item values need to be compared to one another, for a general guide their values can be thought of as 1 of 10 going up, or 10 of 1 going down. EG: 10 bodyguards (each with an item value of ‘4’) might equate to the value of a Private Security Force (item value 5.) This is an example, but we recommend players use this mechanic as a guide rather than a means of precise valuation.

This means that the third commerce point is essentially worth ten level two commerce points and the fifth commerce point is worth one hundred level three points. The narrator needs to take this into account when awarding commerce points as going from 2 to 3 commerce points is infinitely easier than going from 5 to 6.

\section{Assigning Commerce Points} \label{sec:assigning_commerce_points}
At the conclusion of character creation, the Narrator should assign a commerce level of Two, Three or Four to the character based on backstory. Characters can gain additional skill points, showing their need to adapt under harsh conditions or display interesting quirks that come along with wealth.

\begin{itemize}
    \item If the character is assigned a commerce level of \textbf{Two}, they gain \textbf{+1 skill point} to spend on their character, but may only spend it in a skill they have not already assigned any points into. (Note, this can be a vocation or vocational skill as yet unfilled).
    \item If the character is assigned a commerce level of \textbf{Three}, they remain as they are.
    \item If the character is assigned a commerce level of \textbf{Four}, they must roll on the \textbf{‘Quirks of Wealth’} table at the end of this section. (Player and Narrator can instead agree upon a backstory appropriate quirk if they desire).
\end{itemize}

At this point, the player may then opt to increase or decrease their commerce level by one point.

\begin{itemize}
    \item If they \textbf{decrease} it, they gain \textbf{+1 skill points to} spend on their character, but may only spend it on a skill they have not already assigned any points into. (Note, this can be a vocation or vocational skill as yet unfilled).
    \item If they \textbf{increase} it, they must roll on the \textbf{‘Quirks of Wealth’} table at the end of this section. (Player and Narrator can instead agree upon a backstory appropriate quirk if they desire)
\end{itemize}

This may only be done once, and the player can instead opt to ignore this step and accept the commerce level the Narrator assigned them. At the end of this process, the player will be left with a commerce level of between ONE and FIVE.

In \textbf{Fastplay} games, the Narrator can of course choose to ignore these recommendations for their game, but it is strongly recommended for an enjoyable game experience for everyone. Gaining resources is often a very important motivator and plot driver in tabletop RPGs, and one especially wealthy player can very quickly render that moot with a high commerce level (six to eight). (For example, a commerce level six player purchasing extravagant weapons and full plate armour for their entire party upon arriving in the first town to no detriment.)

Players will gain commerce points throughout gameplay which will be assigned by the narrator as a reward for quests (as the narrative dictates) either through the flow of narrative (finding treasure) or role-play (selling loot to shopkeepers.)

\section{Quirks of Wealth Table} \label{sec:quirks_of_wealth_table}

If you roll twice due to the narrator assigning you a Commerce Level of four and then choosing to increase your Commerce Level by one, Re-roll the second result if the narrator deems it incompatible or if you roll the same result twice. (EG 1 + 5/6)

\begin{center}
    \begin{xltabular}{\textwidth}{|c|X|X|} 
     \hline
       & \textbf{Quirk of Wealth} \\ 
     \hline
     \textbf{1} & \textbf{Scrooge:} You will not spend a cent on anyone except yourself \\ 
     \textbf{2*} & \textbf{Patron:} You support an impoverished creative, and cannot refuse their requests. \\
     \textbf{3*} & \textbf{Expensive tastes:} There is an aspect of life where you will only accept the absolute best. (Eg: Food, wine, lodgings) \\
     \textbf{4} & \textbf{Philanthropist:} You are extremely charitable to those in need, even to your detriment. \\
     \textbf{5} & \textbf{Loan Shark:} You will spend your money on others, but demand it back with interest. No exceptions. \\
     \textbf{6} & \textbf{Hidden Price:} If you buy something of value for another, you will expect a favour of significance in return. If refused, you will demand your item. \\
     \textbf{7} & \textbf{Tight Purse:} You will only purchase things with an item value of two or less than your commerce level. \\
     \textbf{8*} & \textbf{Blood Money:} Your wealth secretly stems from a dark source that is kept secret.\\
     \textbf{9} & \textbf{Family Ties:} Your money is tied up in a trust or controlled by someone else. All spending is monitored, and spending that would decrease your CL must be approved by them. \\
     \textbf{10} & \textbf{Fame:} Your wealth has brought you fame appropriate to your CL. (Local celebrity, business owner) - But this attention is unwanted, you stand out and people recognize you. Constantly seeking your advice or attention. \\
     \textbf{11} & \textbf{Debt:} Your wealth is balanced against huge credit/debt, but you’re afloat due to your income. Should you lose your income for a month, you immediately lose 2 Commerce Level. \\
     \textbf{12*} & \textbf{Stolen:} Your wealth is in part stolen from someone else who would very much like it back. \\
     \textbf{13} & \textbf{Elitist:} You treat people with lower commerce levels than you as inferior in an obvious manner. \\
     \textbf{14} & \textbf{Ladder Climber:} You are grovelling and overly nice to those with a higher commerce level than you. \\
     \textbf{15} & \textbf{Romantic Gifter:} You are prone to over-spending on romantic interests to win their hearts in over-the-top displays of love. \\
     \textbf{16*} & \textbf{Closed Trade/Unique Market:} Your source of wealth is limited in scope and only renewable in the locale you come from. This could be a niche business, or a trade good only valuable in one area. You treat your CL as two points lower when outside this area. (Scope depends on the game setting and narrator's discretion) \\
     \textbf{17*} & \textbf{Unethical Business:} People know your money stems from a source most don’t like. It isn’t necessarily illegal, but the common folk dislike you for it. \\
     \textbf{18*} & \textbf{Superiors Favour:} Your success hinges on the support and recommendation of a major celebrity, CEO or noble. If they withdraw that support, your business will collapse. \\
     \textbf{19} & \textbf{Risky Investor:} If the potential exists for profit, you will spend money even if it reduces your profit level. Somewhat susceptible to accomplished con artists. \\
     \textbf{20*} & \textbf{Employees:} Your wealth and investments support some employees, you are obligated to ensure their needs are met, but they can perform basic services for you too. The way you treat them will spread fast. \\
     \hline
    \end{xltabular}
\end{center}

\textit{*Narrator defines specifics.}