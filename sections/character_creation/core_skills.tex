Where attributes showcase the raw physical and mental capabilities of a character, skills represent their learned or unlearned abilities.

In \textbf{Fast Play} games, player characters will be given \textbf{12 skill points} they may assign to their character during character creation. Players may gain additional points to spend in this phase by either accepting a Disabling Characteristic (See: “Disabling Characteristics”) or by assigning attribute points to Intelligence (+3 skill points per point in Intelligence.) In \textbf{Campaign Play}, players will also be able to gain skill points as the story progresses. Each point represents an additional die the character receives when making dice checks using the applicable skill.

%image tooltip

Just like with Attributes, when assigning a point simply mark the box with your pencil as shown above. The tapered box on the far left is there to “cross out” if you take a negative in a skill. These skills can have a max of four points spent on each of them, and a player may also take a single ‘negative’ point from one of these skills, and apply the removed point elsewhere. (Note: This may only be done once, you cannot choose multiple negative skills at character creation)

Although this tapered “negative” box is also present in the Attributes section, players may not choose to have a negative in an attribute – it is only there in case a disabling characteristic demands a negative be applied to an attribute.

Every \textbf{skill} has an attribute (or attributes) it is based on. For example, the skill \textbf{“Persuasion”} is based on the Attribute \textbf{“Intelligence”} When rolling a skill check you need to create the pool of dice that you will be rolling for the check. First, add the 3 dice that make up your base roll. Then, count the number of shaded boxes in the skill you are being asked to make a check for, as well as the number of shaded boxes in that skill’s core attribute. This forms your dice pool for any skill check and will be explained in depth in Chapter 2: \textbf{How to Play}.

For Example, your character has \textbf{1 point} in \textbf{Strength} and \textbf{3 points} in \textbf{Athletics} When they perform an athletics check they will begin with their base of\textbf{3D6} add \textbf{1D6} for their \textbf{STR} Attribute, and \textbf{3D6} for their Athletics skill. The result is \textbf{7D6} to roll with:

%image 3

Skills checks may be attempted even if your character has no points spent on that specific skill. To do this, simply form your dice pool as normal, adding your base 3 dice to the number of shaded boxes in the relevant skill (+0 in the case of unskilled) and then adding a die for your skills core attribute.

Skills are divided into two categories:

\begin{itemize}
    \item \textbf{Core Skills:} Strength/Reflex/Intelligence-based skills.
    \item \textbf{Vocational Skills:} Skills which define a character’s chosen path in life, be it a class, a career or a calling. These skills are crafted between the player and narrator and include combat proficiencies.
\end{itemize}

\section{Core Skills in Detail} \label{sec:core_skills_in_detail}

Core skills in Cogent Roleplay are meant to represent the scope of activities that everyday people are capable of attempting with some chance of success. It includes motions of the body and feats of strength, speed and thought that are near-universally used by all creatures. While sometimes more specific interpretations of these skills are relevant (For example, using ride/pilot to drive a car), the same skill could also be used for a beaver attempting to balance on a log as it floated downstream.

It is in this way that Core Skills primarily differ from Vocational Skills, as Vocational Skills cover tasks related to very specific or learned jobs.

Combat will be introduced fully in Chapter 4, but it is worth noting that Core Skills cannot be added to combat rolls. The only way Core Skills can aid combat rolls is through assists (see \textbf{Assists}, CH3, PT2.) Your combat skills will be covered in \textbf{Vocations}, CH3, PT4. %these need to be changed to actual references once these pages are written!

\subsection{Strength-Based Skills} \label{sec:strength_based_skills}

\textbf{Endurance}

Endurance is the physical trait that represents your body’s ability to process toxins and resist disease. Resisting food poisoning, hot and cold weather and pushing on past exhaustion are all domains of endurance. Endurance also makes wearing and fighting in heavy armour easier, which will be covered further in (\textbf{Armour}, Chapter 4, Part 3.) %ref

\begin{displayquote}
    \textbf{Examples:}
    \begin{itemize}
        \item Approaching a raging inferno and pushing through the heat.
        \item Travelling on a long journey with little to no sleep.
        \item Holding your composure in a drinking competition.
    \end{itemize}
\end{displayquote}

\textbf{Athletics}

Athletics covers physical activity largely related to raw speed and strength. Long jump, high jump, sprinting and weightlifting are all covered by Athletics.

\begin{displayquote}
    \textbf{Examples:}
    \begin{itemize}
        \item Attempting to outrun something.
        \item Shouldering a door open.
        \item Carrying heavy weights.
    \end{itemize}
\end{displayquote}

\textbf{Grip}

Grip encompasses hand strength and your ability to hold on to things. Climbing, holding, squeezing, and crushing are all covered by Grip. Grip can also be used by animals and monsters in several ways, an alligator death roll or a T-Rex crushing a fence with its jaw would all fall under Grip.

\begin{displayquote}
    \textbf{Examples:}
    \begin{itemize}
        \item Dangling from a rope.
        \item Climbing the wall of a castle.
        \item Crushing something in your hand.
    \end{itemize}
\end{displayquote}

\textbf{Swim}

Swim covers all situations where characters are attempting to move in liquids. It is also applicable to science fiction settings in Zero gravity environments. Swimming may be more difficult if a character is wearing heavy clothing or resisting conflicts at the same time, at the narrator's discretion. While the skill is quite linear in function, the cost of failure in these situations is more often than not fatal. Uniquely amongst the skills, failing a swim check differs depending on whether you have assigned skill points to it. If you have, narrative failure applies as normal, with your character making no headway, treading water or otherwise. However, if you have not been assigned any points to swim, your character slips underwater and begins drowning.

\begin{displayquote}
    \textbf{Examples:}
    \begin{itemize}
        \item Swimming across a river.
        \item Diving underwater to search for something.
        \item Floating from one solar panel to another while repairing a space station.
    \end{itemize}
\end{displayquote}

\textbf{Throw}

Throw covers the physical task of launching objects from yourself, and successfully receiving them. The ancient arts of shotput and discus are the domain of the Throw skill.

\begin{displayquote}
    \textbf{Examples:}
    \begin{itemize}
        \item Throwing a dwarf from a cliffside to a stone bridge.
        \item Catching a sack of stolen goods dropped from above.
        \item Throwing a grappling hook over a high ledge.
    \end{itemize}
\end{displayquote}

\subsection{Reflex-Based Skills}

\textbf{Perception}

Perception is the art of detecting things through sight, sound, and or scent, it can even cover “gut feelings” or intuition. It is used as the defence against deception, persuasion, and stealth; it will be common for the Narrator to require players to make environmental “Perception Checks” to gauge how much the characters notice regarding current happenings or places, whether obvious or subtle. While intellect can be vital in perception, it is the speed at which one notices things that can turn the tide in one’s favour. It is important to note many types of skills may commonly be used to assist with perception, especially vocational ones.

\begin{displayquote}
    \textbf{Examples:}
    \begin{itemize}
        \item Checking an area for traps.
        \item Finding the right contact at a nightclub (Assist with Infiltration).
        \item Locate a vendor for a particular item in town (Assist with General Knowledge).
    \end{itemize}
\end{displayquote}

\textbf{Acrobatics}

Acrobatics covers movements requiring agility and finesse as well as balance. Parkour, backflips and walking tightropes are all the domain of Acrobatics. Acrobatics is also the skill used for dodging things and avoiding traps or pitfalls.

\begin{displayquote}
    \textbf{Examples:}
    \begin{itemize}
        \item Performing a gymnastics routine.
        \item Diving out of the way of a speeding car.
        \item Weaving through security lasers.
    \end{itemize}
\end{displayquote}

\textbf{Ride/Pilot}

Any time a creature wishes to take passage on something the Ride/Pilot skill can be used. Riding a horse, driving a car/cart or surfing a wave would all be considered Ride/Pilot checks. When the ride skill is chosen the player must choose what type of animal or craft it applies to. The check should only need to be made if an active effort is required to maintain control. A person riding a train would not need to make a Ride/Pilot check, but the driver would. Clinging to a giant as it walks would be a grip check unless you were in control and directing its movement.

\begin{displayquote}
    \textbf{Examples:}
    \begin{itemize}
        \item Piloting a spaceship.
        \item Steering a raft.
        \item Driving a horse and cart.
    \end{itemize}
\end{displayquote}

\textbf{Sleight of Hand}

Activities that require highly dexterous control of the hands are covered by Sleight of Hand. While more obvious tasks such as picking pockets come to mind, the skill also covers things such as weaving a small basket or threading a needle.

\begin{displayquote}
    \textbf{Examples:}
    \begin{itemize}
        \item Taking something small and fiddly apart successfully.
        \item Picking a lock.
        \item Performing a magic card trick.
    \end{itemize}
\end{displayquote}

\textbf{Stealth}

Sneaking, hiding from others and moving with no noise forms the basis of the Stealth skill. Stealth can be used for setting up ambushes and laying traps. Stealth is often countered by Perception, and the two forces of hide and seek clash constantly.

\begin{displayquote}
    \textbf{Examples:}
    \begin{itemize}
        \item Hiding in the bushes awaiting a caravan to rob.
        \item Moving silently on a rooftop and staying low out of sight.
        \item Approaching a predator from downwind to avoid it catching your scent.
    \end{itemize}
\end{displayquote}

\subsection{Intelligence-Based Skills}

\textbf{General Knowledge}

General Knowledge is the art of listening to rumours, recalling information and applying it where it is needed most. It aids in many tasks while being very specifically only usable on non-specialized tasks. A general knowledge check could never allow you to know how to disarm a bomb, but it might be able to tell you that to disarm a bomb you need to cut a specific wire. Which wire? You have no idea, perhaps if you had a Vocational Skill related to explosives you might know more.

\begin{displayquote}
    \textbf{Examples:}
    \begin{itemize}
        \item Getting a general grasp of who holds power in a region.
        \item Knowing what trade goods are usually imported to or exported from an area.
        \item Broadly understanding how to approach a situation, and what skills you might need on your team to be successful.
    \end{itemize}
\end{displayquote}

\textbf{Deception}

Deception is used to mislead others. This can be done in a variety of ways, from complicated disguises to outright lying. Deception also covers forgery and wilfully sending the wrong impressions through body language. An important thing to note regarding deception is that the knowledge of the person you are trying to deceive/persuade should affect the CL of the roll. Even with this caveat, some things are simply impossible to lie about. You could not convince someone you were their wife or husband for example. Instead, succeeding even against impossible odds often leads to the listener believing that YOU believe what you are saying, which can lead to some very unexpected outcomes!

\begin{displayquote}
    \textbf{Examples:}
    \begin{itemize}
        \item Lying about why you are somewhere you aren't supposed to be.
        \item Disguising yourself as someone else.
        \item Faking an emotion - ie. fake crying.
    \end{itemize}
\end{displayquote}

\textbf{Infiltration}

If stealth is the art of going unseen, Infiltration is the art of going unnoticed. Hiding in plain sight or making the right social connections to ease into a niche social circle are equally valid uses of Infiltration. Infiltration also covers the ability to know and track patrol routes and weak points in defences.

\begin{displayquote}
    \textbf{Examples:}
    \begin{itemize}
        \item Slipping your egg into another bird's nest to have it rear your young, unnoticed.
        \item Weaving your way through a social web until you reach the contacts you seek.
        \item Avoiding mapped guard patrols to slip through a gap in their defences.
    \end{itemize}
\end{displayquote}

\textbf{Persuasion}

Persuasion is the art of convincing people of your point of view. Persuasion can be used in innumerable ways, but bartering, charming people and making sound arguments are the primary focus of the skill.

\begin{displayquote}
    \textbf{Examples:}
    \begin{itemize}
        \item Bargaining for a cheaper price on an item.
        \item Convincing the town guard to give you information to help solve a murder.
        \item Inspiring a boost in morale in a defeated force.
    \end{itemize}
\end{displayquote}

\textbf{Survival}

Survival is the skill associated with staying alive in all environments. It covers scavenging for food, basic first aid as well as finding safe places to rest and lighting fires. In the city, survival keeps you from going too close to dangerous gang territory, it lets you know who to avoid, and where to avoid. In the wilderness, it tells you which berries will make you sick, which meat is spoiled and more.

\begin{displayquote}
    \textbf{Examples:}
    \begin{itemize}
        \item Preparing an animal to be cooked.
        \item Splinting an injured limb.
        \item Collecting rainwater using leaves.
    \end{itemize}
\end{displayquote}