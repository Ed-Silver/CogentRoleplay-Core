There are many circumstances in combat where a player would logically receive either a bonus or penalty to their combat roll. Below are some examples of standard combat modifiers and if a player feels they should receive a bonus that is not covered by the following suggestions they are free to request the bonus from the Narrator and the Narrator will make the final choice. Likewise, the Narrator is free to impose a penalty not covered in the following examples if they feel it is applicable.

SEE CHEAT SHEET REFERENCE: CHSH - Combat Modifiers

\section{Dual Wielding} \label{sec:dual_wielding}

Dual wielding does not give an additional combat roll in a round. What it does is grant an additional dice bonus depending on the weight of the second weapon being used. To duel wield the character must have at least one proficiency point in both of the weapons being used to receive any bonus.

\begin{itemize}
    \item Small Secondary Weapon \newline +2D6 to all combat rolls.
    \item Medium Secondary Weapon \newline +1D6 to all combat rolls.
\end{itemize}

\textbf{\textit{PLEASE NOTE:}} If you are dual wielding, your combat roll is made up using the proficiency and/or weapon bonus of your primary weapon, NOT any proficiencies or weapon bonuses for BOTH weapons.

\textit{Example:}

\begin{displayquote}
    Kelvin wields a Rapier and a Dagger. His attack role is made up as such:
    Base of 3D6, plus +1D6 from INT and +1D6 from REF
    2D6 from his two skill points invested in a Medium Sword Proficiency
    +1D6 from his rapier weapon bonus (the bonus for a medium weapon)
    +2D6 for the dagger he is also wielding (this bonus can ONLY be added because Kelvin has at least one point in the Small Weapon proficiency)
    If Kelvin were to be disarmed from his primary weapon in combat, he will lose any dual wielding bonus, and must revert to the dagger as his primary weapon until his rapier is recovered.
\end{displayquote}

\section{Close Combat}

‘Close combat’ means the fight is so close in physical proximity that the character’s bodies are pressed up against each other in a grapple or wrestle. Being in close combat negates the use of larger weapons and thus the weapon bonuses of any weapon, medium and larger, are lost when in close combat.

In this scenario, a character using a medium (or larger) weapon is considered ‘unarmed’ and as such will receive any applicable penalties.

Entering into close combat is a manoeuvre which minimally requires victory level 1.

Injury reduction from armour is not applicable while in close combat.

Additionally, you can use a medium weapon in close combat if your opponent is using a reach weapon.

\section{Circumstantial Modifiers}

\begin{center}
    \begin{xltabular}{\textwidth}{|l l X|} 
        \hline 
        \textbf{Modifier} & \textbf{Combat Roll} & \textbf{Description} \\ 
        \hline
        High Ground & +2D6 & 
            Bonus received through achieving a height advantage against an opponent, IE standing on a table, riding on a mount etc. \\
        Flank & +2D6 & 
            Bonus received when attacking an opponent actively engaged in something else, IE attacking another character, picking a lock, etc. \\
        Staggered & -2D6 & 
            Penalty received if the opponent selects to stagger through a victory level, or if the Narrator applies it, IE sand is thrown in a character's face, or they slip on oil, etc. \\
        Off Guard & * & 
            CL determined by Narrator IE, attacking a soldier from behind may require a successful ‘Stealth’ check, then the attack (which they cannot defend against) will be a CL assigned by the Narrator, which the player will roll against with a normal combat roll (without a flanking bonus). Any wins achieved over the CL count towards the level of victory. Failing the CL commences the next round of combat, where the opponent is then engaged. \\
        Prone & -4D6 & 
            Penalty lasts for one round or until the character readies themselves, unless they are incapacitated/sleeping. In combat, the penalty counts as taking place during the action of getting up while defending. \\
        Unequal Equipment & * & 
            In combat or out of combat, the task will be much more difficult if the character is not equipped to do so. Picking a lock is much more difficult without lock picks but not impossible. A master lock picker might be able to improvise. Killing a person with a stick is much harder than with a sword. If a character is attempting to perform a task without the appropriate tool, the Narrator assigns an appropriate bonus or penalty to the player, which they must apply when rolling against the CL or when in combat. \\
        Distance & * & 
            A thrown axe will receive greater penalties at distance then a longbow. \\
        Charge & +1D6 &
            If there is enough distance between the character and opponent, they may ‘charge’, adding momentum to their attack. \\
        Brace & +2 Wins vs Charge &
            a brace is a move purely to counter a charge and can only be done if they have a fast enough reflex, or a full round before the charge to prepare. A character with a higher reflex will be able to declare their action after a player declares their charge. Bracing uses the opponent’s charge momentum against them, enabling a prepared defender to stagger, trip or injure the charging opponent. \\
        Incapacitated & * & 
            If a character is incapacitated, they are held in place, tied up, or unconscious and therefore cannot fight back or defend themselves. armour levels are disregarded and the CL a character must exceed to obtain ANY victory level against them is CL1. \\
        \hline
    \end{xltabular}
\end{center}

\textit{Example:}

\begin{displayquote}
    A troll charged Marcus with a combat roll of 6D6, +1D6 for the charge.

    Marcus, being aware of this due to his higher Reflex, will brace against the attack with his combat roll of 7D6.

    The troll rolls a total of 4 wins, and Marcus rolls a total of 3 wins, +2 additional wins due to his high reflex allowing him to brace against the charge. If he were slower to react, Marcus may have been injured in the charge, but due to bracing he is now able to either stagger or injure his opponent.

    PLEASE NOTE: You may further reinforce your 'brace' by making the entire action 'defence', gaining a further +2D6, however no victory levels can be achieved in doing so.
\end{displayquote}