As mentioned, a character’s combat roll stands as both offence and defence. If a character is acting offensively against \textbf{several} opponents at once, in one round they are able to engage against the equivalent of half their total combat roll, rounded down. This can mean that larger weapons (which grant a greater bonus to combat rolls) make it easier to defend against, or attack and hit, multiple opponents.

Ranged weapons usually cannot achieve victories against more than one opponent per round (this is dependent on setting and allowances from the Narrator, for example; a fully automatic machine gun can hit multiple opponents in a single round, see Hardcore Ranged Rules, CH6.) However, a ranged combat roll still stands as a defence against more than one incoming attack; indeed any character can defend against a number of opponents in a single round.

Example:

\begin{displayquote}
    Sir Terrik is fighting two goblins. Terrik has a combat roll of 8D6. This also means that Terrik is capable of defending against, and achieving victories against, a maximum of 4 adjacent opponents in a single round.

    The two goblins he has engaged each have combat rolls of 7D6 (calculated from base 3D6, +2D6 REF, +1D6 from medium swords proficiency and +1D6 from their medium weapons.)

    One of the goblins rolls 4 wins and the other rolls 5 wins.

    Terrik rolls 4 wins.

    This means that he matched one of the goblin's rolls but was beaten by the other. The Goblin chooses to inflict a minor injury with its level 1 victory against Terrick.

    The next round Terrick rolls 6 wins with his recalculated combat roll (a total of 7D6, due to his injury inflicting -1D6), and the goblins each roll 3 wins.

    Terrik achieves a level three victory against both goblins and he chooses to inflict a serious injury to each.

    As a serious injury will kill a character if not treated in the near future, the goblins retreat leaving a trail of blood.
\end{displayquote}