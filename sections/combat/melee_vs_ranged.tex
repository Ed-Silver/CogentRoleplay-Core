The advantage with melee is that your single offensive combat roll can inflict injury to several opponents that did not match your combat roll.

The advantage of ranged fighting is that no combat manoeuvre can be performed against you so long as your opponent is not adjacent to you. Melee combatants, no matter how high their combat roll, cannot inflict injury unless they are close enough. One common disadvantage with many ranged weapons is that no matter how high your combat roll, you can only inflict injury on one opponent each round.

Guns and bows are meant to be used at range and don’t perform well in melee combat. If a character using melee weapons wants to engage in combat against an opponent using ranged, the character with the ranged weapon will incur the standard penalty as described in weapon modifiers, but not right away. The character using the melee weapon must close the distance and the narrator will determine how many rounds this will take (it might not take any rounds if the character is close enough.)

In the first round where the Melee combatant gets to attack, the ranged combatant might still not receive the ascribed penalty, if they have a higher reflex.

If the melee combatant has a higher reflex and is close enough they can close the distance to a combatant using a ranged weapon before they can fire. In this instance, the ranged combatant received the appropriate penalty for using ranged weapons in melee combat. Also, the combatant using the ranged weapon can only injure the melee combatant attacking them if they declared that they were aiming for that combatant in the declaration phase (they might have been aiming for another person), for ranged weapons can only achieve victories against one opponent in a round.

If a ranged combatant was aiming for someone else, their attack can still hit whoever they were aiming for with no reduction to their roll, but when comparing their roll against anyone attacking them in melee combat, their combat roll is reduced by the weapon type.