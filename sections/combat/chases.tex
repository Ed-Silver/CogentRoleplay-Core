Chases are far more common than you might think. Just watch any action movie and see if there’s a chase scene (there usually is.)

In Cogent, chases begin with the Narrator setting a distance level out of 3 categories for the characters in the chase, those being a close distance, medium distance, or long distance.

At a Close Distance, the characters engaged in the chase are relatively near each other, but not near enough to engage in combat without ranged weapons. Narrators must keep ‘relativity’ in mind when deciding on a level of distance – a Close Distance in a car chase might be considered far away for a chase on foot.

When on foot, the characters use the athletics skill to make chase rolls

When using a vehicle (including mounts) the characters use their ride/pilot skill.

From the initial distance set by the narrator, the players roll against each other using the applicable skill (athletics or ride/pilot.) Please note that other skills, vocations and proficiencies can be used to assist the chase role, such as using acrobatics to jump and weave through obstacles to lose your pursuer.

If the pursuer wins the chase roll, the distance between the characters is reduced. If a pursuer wins the chase roll against the target at a close distance, the pursuer ‘catches’ those trying to flee.

If the characters trying to flee win the chase roll, the distance between them and their pursuers increases until a successful roll at a long distance, whereupon the characters have successfully fled.

When the distance is closed, the target trying to flee were literally stopped, IE their horse can be tripped over, dismounting the target, or their car was run into a tree or wall, damaged so it no longer functions – or they ran into a corner with no easy escape or they were tripped up and are now prone.

The level of severity in how the characters were stopped will depend on how severely they failed the final chase roll. The narrator might even use a destiny roll to determine how bad.

Narrators should use chase scenes as an opportunity to bring excitement to the narrative. Be careful not to let chase scenes become a simple turn-by-turn scenario of “you get closer” or “you get farther away”. Instead, describe the obstacles, and make them difficult and exciting.