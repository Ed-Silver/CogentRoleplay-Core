Combat is run in a Round by Round format. Each character can perform 1 action within a round (usually a combat roll) with liberties in regard to the narrative flow of the game. For instance, it doesn’t take an action to shout out something to another character while fighting a monster, though it would take a round to explain something complicated.

The narrator must not time a player’s dialogue to make sure it falls within restrictive timing. Remember narrative flow will make things run far smoother if you are a little relaxed in regard to things like this, while still having limits – you wouldn’t want to let a player pass on huge amounts of information in a single round, in the middle of a battlefield. In those cases, you might say something like, “it will take several rounds to explain everything you want”.

Round by Round combat isn’t played using any grids, timers, boards or miniatures, though sometimes a rough map or drawing of a room or location may help. There isn’t a strict distance character can move within a round, this is left up to the Narrator’s and the player’s own reasoning. A player with a higher REF would be able to run faster and farther than another player with less, and it doesn’t take that much thought to figure out how far a person can run within a round.

Actions in combat rounds are not solely movements and attacks. A combat action can really be anything a person could do within a few seconds, such as turning over a table, swinging from a chandelier, or dropping a boulder onto an enemy. The Narrative flow you develop while playing will guide you in what the players can do in-game, and how long it will take to do it.

\section{Round Order} \label{sec:round_order}

Every action within Round by Round conflicts happens almost at the same time or in direct succession of one another. This means that one character’s action is NOT explicitly followed by another character’s action. It should be considered as everything happening together.

To make sense of this potential ‘chaos’, there is a declaration phase and a performance phase.

SEE CHEAT SHEET REFERENCE: CHSH - Combat Priority Guide

\subsection{Declaration Phase} \label{subsec:declaration_phase}

The Character with the lowest reflex declares what they are doing first. That might seem odd as you could think characters with higher REF should get to move first – but this is the declaration phase – no one is moving yet. If two Characters have the same REF, they each roll 1D6, where the highest resulting roll gets to declare later for the rest of the combat encounter.

Players who declare LATER hold an ADVANTAGE in combat.

Once a player has declared what they are going to perform, they cannot change their mind. They are locked into performing that action in this round. The players who declare after another can choose an action that counters or complements the declared action of another character.

This means that characters with the highest Reflex can see or react to what their opponents are going to do.

Example:

\begin{displayquote}
    Leon, a player character, is duelling Dartane, a non-player character. It's a new combat round and Dartain has a lower Reflex than Leon so he must declare his action first.

    Dartain declares that he intends to jump on top of a table to get a height advantage against Leon, readying for his attack on the next Round. Leon, knowing Dartain's intent, declares that he wishes to knock the table over whilst Dartain jumps on top of it.
\end{displayquote}

\subsection{Performance Phase} \label{subsec:performance_phase}

Once all characters have declared their action, they make their rolls. Though things mostly happen simultaneously within the round, it’s sometimes important to know which blade landed before another, or who managed to perform their action a little bit faster than another. In these situations, the character with the higher reflex will always win.

Example:

\begin{displayquote}
    The Narrator decides that it will be a standard combat action for Leon to knock over the table but Leon won't receive his weapon bonus for this action as he is using a rapier which isn't suited for knocking over tables (if Leon was using something heavy like a greataxe, it could be used to smash down on the side of the table to knock it over and the weapon bonus might be appropriate.)

    Leon's roll is a base of 3D6, +2D6 for his points in Reflex, making 5 dice to roll with in total.

    The player controlling Leon asks the narrator if he can apply his acrobatics skill. The narrator agrees, giving Leon a further +2D6 (7D6 total).

    Leon rolls his 7 dice, resulting in 4 wins. These 4 wins become the CL his opponent, Dartain, must match in order to succeed his declared action and remain on his feet.

    The Narrator instructs Dartain to make an acrobatics check to jump on the table. Under normal circumstances (unopposed) this would have been easy, requiring a CL of 1, but now Dartain must make an acrobatics check against CL 4, quite difficult for a skill check.

    If Dartain succeeds in his action, he will have an advantage in the next round of combat (higher ground), but if he fails, he will have a disadvantage (staggered, or prone, depending on the roll, see Combat Modifiers).
\end{displayquote}

Where “declaration” happens in reverse (slowest players declare first), during “performance”, the order is reversed, and players with the fastest Reflex in that encounter act out their intentions first.